\documentclass[fleqn]{article}
\usepackage[left=1in, right=1in, top=1in, bottom=1in]{geometry}
\usepackage{mathexam}
\usepackage{amsmath}
\usepackage{multicol}
\usepackage{textcomp}
\usepackage[utf8x]{inputenc}
\usepackage{pgfplots}
\usepackage{wasysym} 
\usepackage{tikz}
\pgfplotsset{width=7cm,compat=1.11}

\pagestyle{myheadings}

\textheight=26cm
\textwidth=16cm
\topmargin=-1.0cm
\oddsidemargin=0cm
\parindent=0mm

\input mycom.tex

\let\ds\displaystyle

\begin{document}

\pagenumbering{arabic}
\setcounter{page}{2} 

\vspace{-1.5cm}
{\bf I PARTE. A. Selección Única.} Cada una de las siguientes preguntas tiene una opción correcta, debe marcar con una equis dentro del paréntesis (X). Total $15$ puntos, $1$ punto cada acierto.
\vs


\benu
\item El Máximo Factor Común del polinomio $125a^4b^3c^4-25a^2b^2c^5-20a^5k^2c^3$ corresponde a
\vs

\benu
\item[] (  ) $5a^2c^3$
\item[] (  ) $10a^2b^2c^3$
\item[] (  ) $5a^2b^2k^2c^3$
\item[] (  ) $10a^5b^3k^2c^3$
\eenu
\vs

\item Al factorizar $x^{120}y^{25}+x^{60}y^{75}$, uno de los factores resulta
\vs

\benu
\item[] (  ) $y^{75}$
\item[] (  ) $x^{120}$
\item[] (  ) $x^{120}y^{75}$
\item[] (  ) $x^{60}+y^{25}$
\eenu
\vs

\item ¿Cuántos factores tiene la factorización de $1-m^4$?
\vs

\benu
 
\item[] (  ) seis
\item[] (  ) tres
\item[] (  ) cinco
\item[] (  ) cuatro
\eenu
\vs

\item Uno de los factores de $3ay^2+2xy^2-12ab-8xb$ es
\vs

\benu
\item[] (  ) $y-2b$
\item[] (  ) $y^2+4b$
\item[] (  ) $3a+2x$
\item[] (  ) $2a-3x$
\eenu
\vs

\item La factorización completa de $k^2-10k+25$ es
\vs

\benu
\item[] (  ) $k-5$
\item[] (  ) $(k+5)^2$
\item[] (  ) $(k-5)^2$
\item[] (  ) $(k+5)(k-5)$
\eenu
\vs

%\addtocounter{page}{0}
\pagebreak


\item La factorización completa de $y^4+3y^2-2y-6$ es
\vs

\benu
\item[] (  ) $y-3$
\item[] (  ) $y^2-2$
\item[] (  ) $(y-2)^2$
\item[] (  ) $(y+3)^2$
\eenu
\vs\vs

\item Uno de los factores de $x^3+2x^2-3x-6$ es
\vs

\benu
\item[] (  ) $x-2$
\item[] (  ) $x-3$
\item[] (  ) $x+2$
\item[] (  ) $x^2+3$
\eenu
\vs

\item Un factor del polinomio $-3p+p^2-40$ corresponde a
\vs

\benu
\item[] (  ) $p-5$
\item[] (  ) $p-4$
\item[] (  ) $p+8$
\item[] (  ) $p+5$
\eenu
\vs

\item Al factorizar completamente $9y^2+16x^2-24xy$ resulta
\vs

\benu
\item[] (  ) $(3y+4x)$
\item[] (  ) $(3y+4x)^2$
\item[] (  ) $(3y-4x)^2$
\item[] (  ) $(3y+4x)(3y-4x)$
\eenu
\vs

\item Uno de los factores de $2a^2b-24ab+72b$ corresponde a
\vs

\benu
\item[] (  ) $2b^3$
\item[] (  ) $2a^2b$
\item[] (  ) $a-6$
\item[] (  ) $a+6$
\eenu
\vs\vs

\pagebreak

\item La factorización completa de $a^2-3ab-5a+15b$ corresponde a
\vs

\benu
\item[] (  ) $(a-3b)(a+5)$
\item[] (  ) $(3b-a)(a+5)$
\item[] (  ) $(a-5)(a-3b)$
\item[] (  ) $(a+3b)(a-5)$
\eenu
\vs

\item Al factorizar completamente $mn^4+9m^3n^2$, uno de los factores es
\vs

\benu
\item[] (  ) $n-3m$
\item[] (  ) $n-9m$
\item[] (  ) $n^2+9m^2$
\item[] (  ) $n^3+3m^2$
\eenu
\vs\vs

\item Al factorizar $81-h^4$ uno de los factores es
\vs

\benu
\item[] (  ) $9-h$
\item[] (  ) $3+h$
\item[] (  ) $(3-h)^2$
\item[] (  ) $(3+h)^2$
\eenu
\vs\vs

\item Al factorizar $6b-3a^2-6a+3b^2$, uno de los factores resulta
\vs

\benu
\item[] (  ) $a+b$
\item[] (  ) $b-a$
\item[] (  ) $a-b-2$
\item[] (  ) $a+b-2$
\eenu
\vs\vs

\item Al factorizar completamente $6g^4+g^2-5$, uno de los factores resulta 
\vs

\benu
\item[] (  ) $g^2+1$
\item[] (  ) $g^2+5$
\item[] (  ) $2g^2-1$
\item[] (  ) $3g^2-5$
\eenu
\vs

\eenu

\pagebreak

{\bf B. Correspondencia.} En la columna A se presenta una serie de factorizaciones sucedidas con un paréntesis. En la columna B se presentan los respectivos polinomios precedidos por un símbolo. Factorice cada polinomio de la columna B y relacione lo obtenido con la columna A utilizando dentro de cada paréntesis el símbolo correspondiente. No sobran opciones. (Total $5$ puntos, $1$ punto cada acierto).
\vs\vs

\begin{multicols}{3}

{\hspace{10mm}\bf Columna A}
\vs\vs

\benu
\item[] $(x-6)(x+6)$
\vs
\item[] $(x+3)(x+7)$
\vs
\item[] $(x-5)(x-3)$
\vs
\item[] $(x-10)(x-8)$
\vs
\item[] $(2x-3)(2x+3)$
\vs
\eenu
 
{\hspace{15mm}}
\vs\vs

\benu
\item[] $(\,\,\,\,\,\,)$
\vs
\item[] $(\,\,\,\,\,\,)$
\vs
\item[] $(\,\,\,\,\,\,)$
\vs
\item[] $(\,\,\,\,\,\,)$
\vs
\item[] $(\,\,\,\,\,\,)$
\vs
\eenu

{\hspace{15mm}\bf Columna B}
\vs

\hspace{5mm}

\benu
\item[{\boldmath $\nabla)$}] $x^2-36$
\vs
\item[\hp{\boldmath $\Delta)$}] $4x^2-9$
\vs
\item[{\boldmath $\oplus)$}] $x^2-8x+15$
\vs
\item[{\boldmath $\sum)$}] $x^2-18x+80$
\vs
\item[{\boldmath $\infty)$}] $x^2+10x+21$
\eenu

\end{multicols}

\vs\vs

{\bf C. Respuesta Breve.} Escriba en el espacio en blanco la expresión correcta para que cada una de las siguientes igualdades sea verdadera. Sea cuidadoso en sus respuestas. (Total $5$ puntos, $1$ punto cada respuesta correcta)

\benu
\vs
\vs
\item[{\boldmath $a)$}] $x^2-4x-12 =(x-6)\cdot\fv$
\vs
\item[{\boldmath $b)$}] $y^2-y-6 =\fv\cdot(y+2)$
\vs
\item[{\boldmath $c)$}] $3x^2-5x+2 =(x-1)\cdot\fv$
\vs
\item[{\boldmath $d)$}] $x^2+2x-24 =\fv\cdot(x+6)$
\vs
\item[{\boldmath $e)$}] $23y+12+10y^2 =(2y+3)\cdot\fv$
\eenu


\pagebreak

{\bf II PARTE. Desarrollo.} Resuelva los siguientes problemas con orden y claridad. Deben aparecer todos los procedimientos que justifican la respuesta, en el espacio indicado. (Total $15$ puntos, $5$ puntos cada uno). \\

1) Un topógrafo desea conocer las dimensiones de un terreno rectangular, pero solo conoce su área, que está expresada por el polinomio $81-4m^2+40mn-100n^2$. ¿Cuáles son las dimensiones del terreno en términos de $n$ y $m$, que busca el topógrafo? (Total $5$ puntos)

\vs\vs\vs\vs\vs\vs\vs\vs\vs\vs\vs\vs

2) Factorice {\bf completamente} los siguientes polinomios. (5 puntos cada uno).

\benu
\item [A)] $2wy^2-y^2-2wk^2+k^2$

\vs\vs\vs\vs\vs\vs\vs\vs\vs\vs\vs\vs\vs\vs

\item[B)] $8x^2n^3-2x^2np^2+4x^2n^2p-x^2p^3$
\eenu

\end{document}