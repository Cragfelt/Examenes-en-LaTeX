\documentclass[12pt, fleqn]{article}
\usepackage[left=1in, right=1in, top=1in, bottom=1in]{geometry}
\usepackage{mathexam}
\usepackage{amsmath}
\usepackage{multicol}
\usepackage{textcomp}
\usepackage[utf8x]{inputenc}
\usepackage{pgfplots}
\usepackage{wasysym}
\usepackage{graphicx}
\usepackage{tcolorbox}
\usepackage{pdfpages}

\usepackage{pstricks-add}
\usepackage{pst-solides3d}

\pgfplotsset{width=7cm,compat=1.11}
\usepackage{pgf,tikz}
\usepackage{mathrsfs}
\usetikzlibrary{arrows}

\pagestyle{myheadings}

\textheight=26cm
\textwidth=17cm
\topmargin=-1.0cm
\oddsidemargin=0cm
\parindent=0mm

\input mycom.tex

\let\ds\displaystyle

\begin{document}

\pagenumbering{arabic}
\setcounter{page}{2} 

\vspace{-1.5cm}


\sf

{\bf I PARTE. A. Selección Única.} Cada una de las siguientes preguntas tiene una opción correcta, debe marcar con una equis dentro del paréntesis (X). (Total $12$ puntos, $1$ punto cada acierto).
%\vp

\benu
\item El término que debe sumarse y restarse para completar el cuadrado de $\,\,x^2-4x-11\,\,$ es
\vp

\benu
\item[] \opc $2$. \vf
\item[] \opc $4$ \vf
\item[] \opc $-2$ \vf
\item[] \opc $-4$
\eenu
\vs

\item Al completar el cuadrado de $\,\,3x^2+18x-5\,\,$ se debe sumar y restar el término
\vp

\benu
\item[] \opc $6$\vf
\item[] \opc $9$\vf
\item[] \opc $18$\vf
\item[] \opc $324$
\eenu
\vs

\item El término que completa el cuadrado de $\,\,x^2-10x+7\,\,$ corresponde a \vp

\benu
\item[] \opc $5$
\item[] \opc $25$
\item[] \opc $100$
\item[] \opc $-10$
\eenu
\vs

\item Al completar el cuadrado de $x^2-12x+5$ se obtiene \vp

\benu
\item[] \opc $(x-6)^2+31$
\item[] \opc $(x-6)^2-31$
\item[] \opc $(x+6)^2+41$
\item[] \opc $(x+6)^2-41$
\eenu
\vs

\pagebreak

\item Al completar el cuadrado de $x^2+10x-9$ se obtiene \vp

\benu
\item[] \opc $(x-5)^2-16$ \vp
\item[] \opc $(x-5)^2+16$ \vp
\item[] \opc $(x+5)^2+34$ \vp
\item[] \opc $(x+5)^2-34$
\eenu
\vs

\item Al completar el cuadrado del trinomio $3x^2-6x+12$ se obtiene \vp

\benu
\item[] \opc $(x+3)^2+3$ \vp
\item[] \opc $(x-3)^2-12$ \vp
\item[] \opc $3(x+1)^2+3$ \vp
\item[] \opc $3(x-1)^2+9$
\eenu
\vs

\item La expresión $\dis\frac{24mn-9m^2n}{3mn}$ es equivalente a \vp

\benu
\item[] \opc $8-3m$ \vp
\item[] \opc $8mn^2-3n$ \vp
\item[] \opc $8mn-3m^2n$ \vp
\item[] \opc $8m^2n^3-3mn^2$
\eenu\vs

\item La expresión $\dis\frac{8a^2x^3-10a^3x}{2a^2x}$ es equivalente a \vp

\benu
\item[] \opc $4x^2-5a$ \vp
\item[] \opc $6x^2-8a$ \vp
\item[] \opc $4a^4x^4-5a^5x^2$ \vp
\item[] \opc $6a^4x^4-8a^5x^2$
\eenu\vs

\pagebreak

\item La expresión $(6y+4y^4-8y^2)\div(-2y)$ es equivalente a \vp

\benu
\item[] \opc $10y+4y^4$ \vp
\item[] \opc $-3-2y^3+4y$ \vp
\item[] \opc $-3-4y^4+8y^2$ \vp
\item[] \opc $-3y^2-2y^5+4y^3$
\eenu\vs

\item Al efectuar la división $(25a^8b^6-15a^4b^4+10a^4b^2)\div (5a^4b^2)$ se obtiene
\vp

\benu
\item[] \opc $5a^4b^4-3b^2$ \vp
\item[] \opc $5a^4b^4-3b^2+2$ \vp
\item[] \opc $5a^2b^3-3ab^2+2$ \vp
\item[] \opc $5a^{12}b^8-3a^8b^8+2a^8b^4$
\eenu\vs

\item Al efectuar la división $(30a^4b^4-24a^2b^2+18ab) \div (3a^3b^3)$ resulta
\vp

\benu
\item[] \opc $10a^2b^2-8+6ab$ \vp
\item[] \opc $10a^2b^2+8-6ab$ \vp
\item[] \opc $10ab-8+6a^{-2}b^{-2}$ \vp
\item[] \opc $10ab+8-6a^{-2}b^{-2}$
\eenu
\vs

\item Al dividir $(8x^4y^5-12x^3y^4-4x^5y^4)\div (-4x^5y^4)$ se obtiene
\vp

\benu
\item[] \opc $2xy^{-1}-3x+1$ \vp
\item[] \opc $-2xy^{-1}+3x-1$ \vp
\item[] \opc $2x^{-1}y-3x^{-2}+1$ \vp
\item[] \opc $-2x^{-1}y+3x^{-2}-1$
\eenu
\eenu

\pagebreak

{\bf II PARTE. Desarrollo.} Resuelva los siguientes ejercicios con orden y claridad. Deben aparecer todos los procedimientos que justifican la respuesta, sea ordenado. (Total 12 puntos). \vp

{\bf 1)} Complete el cuadrado del trinomio $\,\,9y^2-18y+36\,\,$ \hfill {\bf (6 puntos)}

\vs\vs\vs\vs\vs\vs\vs\vs\vs\vs\vs\vs\vs\vs\vs\vs\vs\vs

{\bf 2)} Realice las siguientes divisiones 

\benu
\item $\dis\frac{36a^4b^3+12a^3b-24a^2b^4}{-4a^3b^2}$ \hfill {\bf (3 puntos)}

\vs\vs\vs\vs\vs\vs\vs\vs\vs

\item $(45y^5-27xy^3+18x^2y^4)\div(9y^2)$ \hfill {\bf (3 puntos)}
\eenu

\end{document}