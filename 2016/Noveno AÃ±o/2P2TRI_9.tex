\documentclass[fleqn]{article}
\usepackage[left=1in, right=1in, top=1in, bottom=1in]{geometry}
\usepackage{mathexam}
\usepackage{amsmath}
\usepackage{multicol}
\usepackage{textcomp}
\usepackage[utf8x]{inputenc}
\usepackage{pgfplots}
\usepackage{wasysym} 
\usepackage{tikz}
\pgfplotsset{width=7cm,compat=1.11}

\pagestyle{myheadings}

\textheight=28cm
\textwidth=16cm
\topmargin=-2.5cm
\oddsidemargin=0cm
\parindent=0mm

\input mycom.tex

\let\ds\displaystyle


\begin{document}

\bc {\huge\sc Colegio Salesiano Don Bosco} \ec \vspace{-5mm}
\bc {\Large\sc Sétimo Año} \ec \vspace{-5mm}
\hrulefill \vspace{2mm}

\normalsize\sf
{\sf Conceptos Básicos y Ángulos \hfill Ejercicios\,\#2 \\
Sétimo Año \hfill Prof. Luis Diego Aguilar S.}

\hrulefill \vspace{2mm}

Escriba en el rectángulo una {\bf V} si es {\sc Verdadera} o una {\bf F} si es {\sc Falsa} en cada uno de los recuadros correspondientes a cada proposición:

\benu

\item Un segmento tiene longitud. \hfill \fv
\item Un ángulo llano es igual a dos rectos. \hfill \fv
%\item Una recta posee solo dos semirrectas. \hfill \fv
\item Una recta divide al plano en dos semiplanos. \hfill \fv
%\item Un ángulo completo es congruente a uno nulo. \hfill \fv
\item Tres puntos no colineales determinan un plano. \hfill \fv
\item Una recta posee infinita cantidad de segmentos. \hfill \fv
\item Dos rectas oblicuas también son perpendiculares. \hfill \fv
\item Dos rectas perpendiculares también son secantes. \hfill \fv
\item Un ángulo nulo y uno recto son complementarios. \hfill \fv
\item Por un punto cruza una infinita cantidad de rectas. \hfill \fv
\item La intersección de dos planos paralelos es una recta. \hfill \fv
\item Por una recta cruza una infinita cantidad de planos. \hfill \fv
\item Dos ángulos adyacentes son también suplementarios. \hfill \fv
\item Dos rectas perpendiculares forman cuatro ángulos rectos. \hfill \fv
\item Dos rectas paralelas se intersecan en un punto en el infinito. \hfill \fv
\item Cualesquiera ángulos opuestos por el vértice son congruentes. \hfill \fv
\item Un ángulo obtuso y uno agudo siempre son sumplementarios. \hfill \fv
\item La bisectriz de un ángulo llano lo divide en dos ángulos rectos. \hfill \fv
\item Dos o más ángulos son congruentes si sus medidas suman $180^o$.. \hfill \fv
\item Una bisectriz divide a un ángulo en dos ángulos complementarios. \hfill \fv
\item Una semirrecta es un rayo que no contiene a su origen o frontera. \hfill \fv
\item Dos ángulos son consecutivos si tienen un lado y un vértice común. \hfill \fv
\item Dos o más ángulos son complementarios si sus medidas suman $90^o$. \hfill \fv
\item Dos rectas se llaman alabeadas cuando no son coplanares y no se intersecan. \hfill \fv
\item El lado común de dos ángulos consecutivos es la bisectriz del ángulo que los comprende. \hfill \fv

\eenu

\end{document}