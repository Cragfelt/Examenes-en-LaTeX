\documentclass[12pt, fleqn]{article}
\usepackage[left=1in, right=1in, top=1in, bottom=1in]{geometry}
\usepackage{mathexam}
\usepackage{amsmath}
\usepackage{multicol}
\usepackage{textcomp}
\usepackage[utf8x]{inputenc}
\usepackage{pgfplots}
\usepackage{tcolorbox}
\usepackage{wasysym} 
\usepackage{tikz}
\pgfplotsset{width=7cm,compat=1.11}

\pagestyle{myheadings}

\textheight=26cm
\textwidth=16.5cm
\topmargin=-1.25cm
\oddsidemargin=0cm
\parindent=0mm

\input mycom.tex

\let\ds\displaystyle

\begin{document}

\sf

\pagenumbering{arabic}
\setcounter{page}{2} 

\vspace{-2cm}
{\bf I PARTE. A. Selección Única.} Cada una de las siguientes preguntas tiene una opción correcta, debe marcar con una equis dentro del paréntesis {\bf (X)}. {\bf Total 20 puntos, 1 punto cada acierto.}
\vp

\benu
\item Al completar el cuadrado de  $x^2-6x+5$ se obtiene \vp

\benu
\item[] \opc $(x-3)^2-4$
\item[] \opc $(x-3)^2+4$
\item[] \opc $(x+3)^2+4$
\item[] \opc $(x+3)^2-4$
\eenu
\vs

\item Al completar el cuadrado de  $x^2-9x+5$ se obtiene \vp

\benu
\item[] \opc $\dis\iz x-\frac92\der^2-\frac{71}{2}$
\item[] \opc $\dis\iz x+\frac92\der^2-\frac{71}{2}$
\item[] \opc $\dis\iz x-\frac92\der^2-\frac{91}{2}$
\item[] \opc $\dis\iz x+\frac92\der^2-\frac{91}{2}$
\eenu
\vs

\item Al completar el cuadrado de  $4x^2-8x+1$ se obtiene \vp

\benu
\item[] \opc $4(x-1)^2-3$
\item[] \opc $4(x+1)^2-3$
\item[] \opc $4(x-1)^2-5$
\item[] \opc $4(x+1)^2-5$
\eenu
\vs

\item Si se completa el cuadrado de $3x^2+18x-7$ resulta obtiene \vp

\benu
\item[] \opc $3(x+3)^2+34$
\item[] \opc $3(x+3)^2-34$
\item[] \opc $3(x+3)^2+20$
\item[] \opc $3(x+3)^2-20$
\eenu
\vs

\pagebreak

\item Al efectuar la división $(40m^3n^3-8m^2n^2+20mn) \div (2m^2n^3)$ resulta
\vp

\benu
\item[] \opc $20m^2n-4mn+10mn^2$
\item[] \opc $20m^5n^6-4m^4n^5+10m^3n^4$
\item[] \opc $20m^2+4mn^{-1}-10m^{-1}n^{-2}$
\item[] \opc $20m^2-4mn^{-1}+10m^{-1}n^{-2}$
\eenu
\vs

\item Al dividir $(8x^4y^5-12x^3y^4-4x^5y^4)\div (-4x^5y^4)$ se obtiene
\vp

\benu
\item[] \opc $2xy^2-3y+x^2y$
\item[] \opc $2xy^2-3y-x^2y$
\item[] \opc $-2xy^2+3y+x^2y$
\item[] \opc $-2xy^2+3y-x^2y$
\eenu
\vp

\item Al efectuar la división $(15a^8b^6-10a^4b^4+5a^4b^2)\div (5a^4b^2)$ se obtiene
\vp

\benu
\item[] \opc $3a^4b^4-5b^2$
\item[] \opc $3a^4b^4-2b^2+1$
\item[] \opc $3a^2b^3-2ab^2+1$
\item[] \opc $3a^{12}b^8-2a^8b^8+x^8b^4$
\eenu
\vspace{-4mm}
´
\item Si $C(x)$ es el cociente y $R(x)$ el residuo de dividir $P(x)$ entre $A(x)$, entonces se puede afirmar con certeza que \vp

\benu
\item[] \opc $R(x)=P(x)\cdot A(x)+C(x)$
\item[] \opc $R(x)=P(x)\cdot C(x)+A(x)$
\item[] \opc $P(x)=R(x)\cdot A(x)+C(x)$
\item[] \opc $P(x)=C(x)\cdot A(x)+R(x)$
\eenu
\vp

\item El residuo de $(x^3+x-5+2x^2)\div(x+2)$ es
\vp

\benu
\item[] \opc $9$
\item[] \opc $13$
\item[] \opc $-3$
\item[] \opc $-7$
\eenu

\pagebreak

\item El cociente de $(x^3+3x-4x^2)\div(x-3)$ corresponde a \vp

\benu
\item[] \opc $6x$
\item[] \opc $x^2+6$
\item[] \opc $x^2+x$
\item[] \opc $x^2-x$
\eenu
\vs

\item Al simplificar al máximo la expresión $\displaystyle\frac{2x^2+5x-3}{2x^2-7x+3}$, resulta
\vp

\benu
\item[] \opc $\dis\frac{x+3}{x-3}$ \vp
\item[] \opc $\dis\frac{x-3}{x+3}$ \vp
\item[] \opc $\dis\frac{2x-3}{2x+3}$ \vp
\item[] \opc $\dis\frac{2x+3}{2x-3}$
\eenu
\vs

\item La expresión $\dis\frac{4y^2-12y+9}{4y^2-9}$ simplificada al máximo, es equivalente a
\vp

\benu
\item[] \opc $1$
\item[] \opc $-1$
\item[] \opc $\dis\frac{2y-3}{2y+3}$
\item[] \opc $\dis\frac{2y+3}{2y-3}$
\eenu
\vp

\item Al multiplicar $\dis\frac{xy-2y}{x^2y^2-4y^2}\cdot\dis\frac{x^2y-4xy+y}{yx+2y}$ se obtiene
\vp

\benu
\item[] \opc $y-2x$
\item[] \opc $y^2+4x$
\item[] \opc $3y+2x$
\item[] \opc $2y-3x$
\eenu
\vs

\pagebreak

\item La expresión $\dis\frac{x^2-7x+6}{x^2-4x-12}$ es equivalente a
\vp

\benu
\item[] \opc $\dis\frac{x-2}{x+4}$ \vp
\item[] \opc $\dis\frac{x-1}{x+2}$ \vp
\item[] \opc $\dis\frac{x-1}{x-2}$ \vp
\item[] \opc $\dis\frac{x-2}{x-4}$
\eenu
\vs

\item Al simplificar al máximo la expresión $\dis\frac{4a^2-8a}{a^3+5a^2-14a}$ resulta
\vp

\benu
\item[] \opc $\dis\frac{4}{a+7}$ \vp
\item[] \opc $\dis\frac{4}{a-7}$ \vp
\item[] \opc $\dis\frac{a-2}{a^2+5a-14}$ \vp
\item[] \opc $\dis\frac{4(a-2)}{(a-7)(a+2)}$
\eenu
\vs

\item Al efectuar la operación $\dis\frac{x^2+2x-3}{x^2-x}\cdot\frac{x^2-3x}{x^2-9}$ se obtiene como resultado
\vp

\benu
\item[] \opc $1$
\item[] \opc $\dis\frac x3$
\item[] \opc $-1$
\item[] \opc $\dis\frac{x+1}{x-1}$
\eenu

\pagebreak

\item ¿Cuál es el resultado de  $\dis\frac{5x^2}{3y^2}\div\frac{25x^3}{9y^2}$?
\benu
\item[] \opc $\dis\frac{5y}{3x}$
\item[] \opc $\dis\frac{5x}{3y}$
\item[] \opc $\dis\frac{3y}{5x}$
\item[] \opc $\dis\frac{3x}{5y}$
\eenu\vs

\item La simplificación máxima de $\dis\frac{2x}{x^2-x}\div\frac{4}{x^2-2x+1}$ es \vp

\benu
\item[] \opc $\dis\frac{x-1}{2}$
\item[] \opc $\dis\frac{2}{x-1}$
\item[] \opc $\dis\frac{x2-6x}{x(x-1)^2}$
\item[] \opc $\dis\frac{8x}{(x^2-x)(x-1)^2}$
\eenu
\vp

\item La expresión $\dis\frac{2}{x+2}\cdot\frac{x^2-4}{2x-4}$ es equivalente a \vp

\benu
\item[] \opc $1$
\item[] \opc $2$
\item[] \opc $x+2$
\item[] \opc $x-2$
\eenu
\vp

\item Al efectuar la operación $\dis\frac{x^2-3x}{x^2-9}\div\frac{x^2-x}{x^2+2x-3}$, se obtiene como resultado
\benu
\item[] \opc $1$
\item[] \opc $\dis\frac 3x$
\item[] \opc $-1$
\item[] \opc $\dis\frac{x+1}{x-1}$
\eenu\vs

\eenu
\pagebreak

%{\bf B. Respuesta Corta.} Los siguientes ejercicios deben ser resueltos en forma concisa y breve. Incluya el procedimiento que justifique su respuesta. Sea ordenado y cuidadoso en sus respuestas. {\bf (Total 10 puntos, 1 punto desarrollo, 1 punto la respuesta correcta).} \vs\vp

%{\bf 1.} Racionalice el numerador o el denominador de cada una de las siguientes expresiones algebraicas, según sea el caso. Simplifique al máximo el resultado. \vs\vs

%\benu
%\item $\dis\frac{3}{4\sqrt{5}}=$ \hrulefill \,\,\, {\bf R/}\compl.

%\vs\vs\vs\vs

%\item $\dis\frac{-2}{\sqrt{10x}}=$ \hrulefill \,\,\, {\bf R/}\compl.

%\vs\vs\vs\vs

%\item $\dis\frac{\sqrt{3xy}}{3x}=$ \hrulefill \,\,\, {\bf R/}\compl.

%\vs\vs\vs\vs

%\item $\dis\frac{-5m}{\sqrt{15mn}}=$ \hrulefill \,\,\, {\bf R/}\compl.

%\vs\vs\vs\vs

%\item $\dis\frac{\sqrt{4ab}}{4}=$ \hrulefill \,\,\, {\bf R/}\compl.

%\vs\vs\vs\vs

%\eenu

%\pagebreak

{\bf II PARTE. Desarrollo.} Resuelva correctamente los siguientes problemas con orden y claridad. Deben aparecer todos los procedimientos que justifican la respuesta, en el espacio indicado.  {\bf (Total 30 puntos).} \vp

{\bf 1)} Complete el cuadrado de $3x^2-2x+7$ \hfill {\bf (5 puntos)}

\vs\vs\vs\vs\vs\vs\vs\vs\vs\vs\vs\vs\vs

{\bf 2)} Simplifique al máximo  $\,\,\dis\frac{3ab-3ax+yb-yx}{3ax+yx+3ab+yb}$ \hfill {\bf (5 puntos)}

\vs\vs\vs\vs\vs\vs\vs\vs\vs\vs\vs\vs\vs

{\bf 3)} Realice la operación $\dis\frac{a^2-b^2}{a^2-2ab+b^2}\cdot\frac{a-b}{ab+a^2}\div\frac{1}{a}$ \hfill {\bf (5 puntos)}

%\vs\vs\vs\vs\vs\vs\vs\vs\vs\vs\vs\vs\vs\vs

\pagebreak

%3) Racionalice el denominador de las siguientes expresiones algebraicas.

%\benu
%\item [A)] $\dis\frac{\sqrt x+1}{1-\sqrt x}$ \hfill {\bf (3 puntos)}

%\vs\vs\vs\vs\vs\vs\vs\vs\vs

%\item[B)] $\dis\frac{a-b}{\sqrt a+\sqrt b}$ \hfill {\bf (3 puntos)}
%\eenu


{\bf 4)} Resuelva el siguiente problema. \hfill {\bf (5 puntos)}
\vs

\begin{tcolorbox}
El área de un rectángulo está representada mediante la expresión
$10x^3-32x^2+6x$. 

Si el ancho del rectángulo está dado por $5x-1$, ¿cuál es el largo del rectángulo?
\end{tcolorbox}

\vs\vs\vs\vs\vs\vs\vs\vs\vs\vs\vs\vs\vs\vs\vs

{\bf 5)} Racionalice el denominador de las siguientes expresiones algebraicas. Simplifique al máximo el resultado.

\benu
\item [A)] $\dis\frac{-2}{\sqrt{10x}}$ \hfill {\bf (3 puntos)}

\vs\vs\vs\vs

\item[B)] $\dis\frac{-5m}{\sqrt{15mn}}$ \hfill {\bf (3 puntos)}

\vs\vs\vs\vs

\item[C)] $\dis\frac{x^2-y^2}{\sqrt{x-y}}$ \hfill {\bf (4 puntos)}
\eenu

\end{document}


\item [A)] $\dis\frac{\sqrt x+1}{1-\sqrt x}$ \hfill {\bf (3 puntos)}