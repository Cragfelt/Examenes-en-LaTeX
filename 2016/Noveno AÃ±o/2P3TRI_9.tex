\documentclass[12pt, fleqn]{article}
\usepackage[left=1in, right=1in, top=1in, bottom=1in]{geometry}
\usepackage{mathexam}
\usepackage{amsmath}
\usepackage{multicol}
\usepackage{textcomp}
\usepackage[utf8x]{inputenc}
\usepackage{pgfplots}
\usepackage{tcolorbox}
\usepackage{wasysym} 
\usepackage{tikz}
\usepackage{array}
\pgfplotsset{width=7cm,compat=1.11}

\pagestyle{myheadings}

\textheight=26cm
\textwidth=16.5cm
\topmargin=-1.25cm
\oddsidemargin=0cm
\parindent=0mm

\input mycom.tex

\let\ds\displaystyle

\begin{document}

\sf

\pagenumbering{arabic}
\setcounter{page}{2} 

\vspace{-2cm}
{\bf I PARTE. A. Selección Única.} Cada una de las siguientes preguntas tiene una opción correcta, debe marcar con una equis dentro del paréntesis {\bf (X)}. {\bf Total 20 puntos, 1 punto cada acierto.}
\vp

\benu
\item Para la ecuación $\,\,2x^2-7x+3=0\,\,$ se cumple que \vp

\benu
\item[] \opc no tiene soluciones reales.
\item[] \opc tiene dos raíces reales distintas.
\item[] \opc tiene únicamente una solución real.
\item[] \opc tiene más de dos soluciones reales distintas.
\eenu
\vs

\item El valor del discriminante de la ecuación  $\,\,3x^2+4x-5=0\,\,$ es \vp

\benu
\item[] \opc $36$
\item[] \opc $76$
\item[] \opc $-4$
\item[] \opc $-44$
\eenu
\vs

\item Para la ecuación $\,\,3x^2-x=-2\,\,$ se cumple que \vp

\benu
\item[] \opc no tiene soluciones reales.
\item[] \opc tiene dos raíces reales distintas.
\item[] \opc tiene únicamente una solución real.
\item[] \opc tiene más de dos soluciones reales distintas.
\eenu
\vs

\item Analice las siguientes proposiciones referentes a la ecuación cuadrática $ax^2+bx+c=0$.
\vp

\begin{tcolorbox}
\bc
\benu
\item[I.] Si $\Delta>0$ entonces no tiene soluciones reales.
\item[II.] Si $\Delta=0$ entonces existe una única solución real.
\item[I.] Si $\Delta>0$ entonces existen dos soluciones reales distintas.
\eenu
\ec
\end{tcolorbox}

¿Cuáles de ellas son siempre {\sc correctas}?

\benu
\item[] \opc Todas.
\item[] \opc Solo II.
\item[] \opc Solo I y II.
\item[] \opc Solo II y III.
\eenu
\vs

\pagebreak

\item Considere las siguientes ecuaciones cuadráticas.
\vp

\begin{tcolorbox}
\bc
\benu
\item[I.] $4x^2-12=0$
\item[II.] $2x-(3x^2+1)=0$
\eenu
\ec
\end{tcolorbox}

¿Cuáles de ellas tienen dos soluciones reales?

\benu
\item[] \opc Ambas.
\item[] \opc Ninguna.
\item[] \opc Solo la I.
\item[] \opc Solo la II.
\eenu
\vs

\item La ecuación $\,\,x^2=2(x-2)\,\,$ se caracteriza porque
\vp

\benu
\item[] \opc no tiene soluciones reales.
\item[] \opc posee dos soluciones reales.
\item[] \opc tiene tres soluciones diferentes.
\item[] \opc posee dos soluciones reales iguales.
\eenu
\vp

\item La ecuación $\,\,x^2-2=0\,\,$ posee
\vp

\benu
\item[] \opc una solución real.
\item[] \opc cero soluciones reales.
\item[] \opc dos soluciones racionales.
\item[] \opc dos soluciones irracionales.
\eenu
´
\item El conjunto solución de la ecuación $x^2=4x-4$ es
\vp

\benu
\item[] \opc \O \vp
\item[] \opc $\{-1,1\}$ \vp
\item[] \opc $\{-1,2\}$ \vp
\item[] \opc $\{-2,1\}$
\eenu
\vs

\pagebreak

\item Considere las siguientes ecuaciones cuadráticas.
\vp

\begin{tcolorbox}
\bc
\benu
\item[I.] $3x^2+16=0$
\item[II.] $3x^2+16x=0$
\eenu
\ec
\end{tcolorbox}

¿Cuáles de ellas tienen soluciones reales?

\benu
\item[] \opc Ambas.
\item[] \opc Ninguna.
\item[] \opc Solo la I.
\item[] \opc Solo la II.
\eenu
\vs

\item Una solución de $2x^2-x-3=0$ es \vp

\benu
\item[] \opc $1$
\item[] \opc $\dis\frac32$
\item[] \opc $\dis\frac{-1}{2}$
\item[] \opc $\dis\frac{-3}{2}$
\eenu
\vs

\item El conjunto solución de la ecuación $x^2=5$ corresponde a
\vp

\benu
\item[] \opc $\{\sqrt6\}$
\item[] \opc $\{-5,5\}$
\item[] \opc $\{-\sqrt5,\sqrt5\}$
\item[] \opc $\{1-\sqrt6,1+\sqrt6\}$
\eenu
\vp

\item Una solución de $2x^2+3x=x^2+2x+12$ es
\vp

\benu
\item[] \opc $3$
\item[] \opc $4$
\item[] \opc $\dis\frac43$
\item[] \opc $\dis-\frac{5}{6}$
\eenu
\vs

\pagebreak

\item El conjunto solución de  $125-5x^2=0$ es
\vp

\benu
\item[] \opc $\{\,\,\}$ \vp
\item[] \opc $\{5\}$ \vp
\item[] \opc $\{0,25\}$ \vp
\item[] \opc $\{-5,5\}$
\eenu
\vs

\item El vértice de la parábola $(x-1)^2=5$ es
\vp

\benu
\item[] \opc $(1,5)$ \vp
\item[] \opc $(-1,5)$ \vp
\item[] \opc $(1,-5)$ \vp
\item[] \opc $(-1,-5)$
\eenu
\vs

\item La gráfica de $f(x)=x^2-4x+3$ interseca al eje $x$ en los puntos \vp

\benu
\item[] \opc $(1,0)$ y $(3,0)$ \vp
\item[] \opc $(0,1)$ y $(0,3)$ \vp
\item[] \opc $(-3,0)$ y $(-1,0)$ \vp
\item[] \opc $(0,-3)$ y $(0,-1)$
\eenu\vs

\item La gráfica de $f(x)=-x^2+6x-8$ interseca al eje $y$ en \vp

\benu
\item[] \opc $(-8,0)$ \vp
\item[] \opc $(0,-8)$ \vp
\item[] \opc $(2,0)$ y $(4,0)$ \vp
\item[] \opc $(0,4)$ y $(0,4)$
\eenu\vs

\pagebreak

\item La gráfica de la función $f(x)=2x^2-3x+4$ \vp

\benu
\item[] \opc no interseca al eje $y$. \vp
\item[] \opc no interseca al eje $x$. \vp
\item[] \opc interseca al eje $y$ en dos puntos. \vp
\item[] \opc interseca al eje $x$ en dos puntos.
\eenu
\vp

\item El eje de simetría de $f(x)=x^2-10x+25$ corresponde a

\benu
\item[] \opc $x=0$ \vf
\item[] \opc $x=10$ \vf
\item[] \opc $x=25$ \vf
\item[] \opc $x=-10$
\eenu\vs

\item El vértice de la parábola $f(x)=x^2+1$ corresponde al punto

\benu
\item[] \opc $(0,1)$ \vf
\item[] \opc $(1,0)$ \vf
\item[] \opc $(-1,0)$ \vf
\item[] \opc $(0,-1)$
\eenu\vs

\item El vértice de la función dada por $f(x)=x-x^2-12$ es \vp

\benu
\item[] \opc $\dis\left(\frac12,\frac{49}{4}\right)$ \vf
\item[] \opc $\dis\left(-\frac12,-\frac{49}{4}\right)$ \vf
\item[] \opc $\dis\left(\frac12,-\frac{47}{4}\right)$ \vf
\item[] \opc $\dis\left(-\frac12,\frac{47}{4}\right)$
\eenu\vs

\eenu
\pagebreak

{\bf B. Respuesta Corta.} Escriba sobre el espacio en blanco la respuesta correcta según sea el caso. Sea ordenado y cuidadoso. (Total 20 puntos, 1 punto cada respuesta correcta).\vs

\benu
\item De acuerdo con los datos de la gráfica de la parábola, escriba la concavidad los puntos solicitados. \vs\vs\vs

\begin{minipage}{\linewidth}
\vspace{-.5cm}
      \begin{minipage}{0.5\linewidth}
        \hspace{-1.5cm}
        \benu
            \item Intersecciones con el eje $x$ \\ \\
            \compl \,\, y \,\,\compl.\vp
            \item Intersección con el eje $y$ \\ \\
            \compl.\vp
            \item Vértice \compl. \vp
            \item Concavidad \compl.
            
        \eenu
      \end{minipage}
      \hspace{-.5cm}
      \begin{minipage}{0.45\linewidth}
         \begin{tikzpicture}[line cap=round,line join=round,x=1.0cm,y=1.0cm,scale=.75]
\draw[->,color=black] (-4.3,0.) -- (6.1,0.);
\foreach \x in {-4,-3,-2,-1,1,2,3,4,5,6}
\draw[shift={(\x,0)},color=black] (0pt,2pt) -- (0pt,-2pt) node[below] {\footnotesize $\x$};
\draw[->,color=black] (0.,-1.94) -- (0.,6.3);
\foreach \y in {-1,1,2,3,4,5,6}
\draw[shift={(0,\y)},color=black] (2pt,0pt) -- (-2pt,0pt) node[left] {\footnotesize $\y$};
\draw[color=black] (0pt,-10pt) node[right] {\footnotesize $0$};
\clip(-4.3,-1.94) rectangle (6.1,6.3);
\draw [samples=50,rotate around={0.:(2.,-1.)},xshift=2.cm,yshift=-1.cm,line width=1.2pt,domain=-4.0:4.0)] plot (\x,{(\x)^2/2/0.5});
\begin{scriptsize}
\draw [fill=black] (0.,3.) circle (2pt);
\draw [fill=black] (3.,0.) circle (2pt);
\draw [fill=black] (1.,0.) circle (2pt);
\draw [fill=black] (2.,-1.) circle (2pt);
\end{scriptsize}
\end{tikzpicture}
      \end{minipage}
\end{minipage}
\vs\vs\vs

\item Para la función dada por $f(x)=x^2-2x-3$ determine: \vs\vs

        \benu
            \item Vértice \compl. \vp
            \item Eje de simetría: \compl.\vp
            \item Intersección con el eje $y$: \compl.\vp
            \item Intersecciones con el eje $x$:
            \compl \,\, y \,\,\compl.\vp
        \eenu
        \vs

\pagebreak

\item Complete la siguiente tabla con la información que se le pide. \vs

\bc
{\setlength{\extrarowheight}{20pt}
\begin{tabular}{|>{\hspace{4pt}}l|l<{\hspace{-2pt}}|}
\hline
\hline
\bf Ecuación & \bf Discriminante \\
\hline
$5x^2+12x-9=0$ &  \\
\hline
$49x^2+81=0$ &  \\
\hline
$x(x+11)=-24$ &  \\
\hline
$x(x+3)=5x+3$ & \\
\hline
$3x^2=12x-5$ & \\
\hline
\hline
\end{tabular}}
\ec
\vs\vs

\item Responda en los espacios vacíos lo que se le pide. \vs

\benu
\item El vértice de la parábola $y=(x-4)^2+1$ es \compl. \\

\item El eje de simetría de $f(x)=9-4x^2$ corresponde a \compl. \\

\item La parábola definida por $y=8x-2x^2+10$ es cóncava hacia \compl. \\

\item La intersección con el eje $y$ de la función $f(x)=(x+2)^2-11$ es \compl. \\

\item ¿Cuántas veces interseca la parábola $y=4x^2-4x+1$ con el eje $x$? \compl.
\eenu
\eenu
\vs

\pagebreak

{\bf II PARTE. Desarrollo.} Resuelva correctamente los siguientes ejercicios con orden y claridad. Deben aparecer todos los procedimientos que justifican la respuesta, en el espacio indicado.  {\bf (Total 20 puntos).} \vp

{\bf 1)} Resuelva las siguientes ecuaciones cuadráticas. \hfill {\bf (4 puntos cada una)}

\vs

\benu
\item $2x(x-2)+2=5-3x$

\vs\vs\vs\vs\vs\vs\vs\vs\vs\vs\vs\vs\vs\vs\vs\vs

\item $(x+5)(2x-1)=x(x+9)$
\eenu

\pagebreak

{\bf 2)} Analice la función cuadrática $f(x)=x^2+2x-15$ \hfill {\bf (7 puntos)}

\benu
\item Vértice (2 puntos).
\item Concavidad (1 punto).
\item Eje de simetría (1 punto).
\item Intersecciones con los ejes (3 puntos).
\eenu

\vs\vs\vs\vs\vs\vs\vs\vs\vs\vs\vs\vs\vs

{\bf 3)} Trace la parábola {\bf cóncava hacia arriba}, según la información de la tabla. {\bf (5 puntos, 1 punto cada par ordenado, 1 punto el trazo de la parábola).} \vs

\begin{tabular}{|c|c|c|c|c|}
\hline
\bf \bm $x$ & $0$ & $-3$ & $2$ & $-1$ \\
\hline
\bm $y$  & $-6$ & $0$ & $0$ & $-7$ \\
\hline
\end{tabular}

\bc
\definecolor{cqcqcq}{rgb}{0.7529411764705882,0.7529411764705882,0.7529411764705882}
\begin{tikzpicture}[line cap=round,line join=round,x=1.0cm,y=1.0cm,scale=.9]
\draw [color=cqcqcq,, xstep=1.0cm,ystep=1.0cm] (-5.72,-7.34) grid (5.8,3.6);
\draw[->,color=black] (-5.72,0.) -- (5.8,0.);
\foreach \x in {-5,-4,-3,-2,-1,1,2,3,4,5}
\draw[shift={(\x,0)},color=black] (0pt,2pt) -- (0pt,-2pt) node[below] {\footnotesize $\x$};
\draw[->,color=black] (0.,-7.34) -- (0.,3.6);
\foreach \y in {-7,-6,-5,-4,-3,-2,-1,1,2,3}
\draw[shift={(0,\y)},color=black] (2pt,0pt) -- (-2pt,0pt) node[left] {\footnotesize $\y$};
\draw[color=black] (0pt,-10pt) node[right] {\footnotesize $0$};
\clip(-5.72,-7.34) rectangle (5.8,3.6);
\end{tikzpicture}
\ec

\end{document}
