\documentclass[12pt, fleqn]{article}
\usepackage[left=1in, right=1in, top=1in, bottom=1in]{geometry}
\usepackage{mathexam}
\usepackage{amsmath}
\usepackage{multicol}
\usepackage{textcomp}
\usepackage[utf8x]{inputenc}
\usepackage{pgfplots}
\usepackage{wasysym}
\usepackage{graphicx}
\usepackage{tcolorbox}
\usepackage{pdfpages}

\usepackage{pstricks-add}
\usepackage{pst-solides3d}

\pgfplotsset{width=7cm,compat=1.11}
\usepackage{pgf,tikz}
\usepackage{mathrsfs}
\usetikzlibrary{arrows}

\pagestyle{myheadings}

\textheight=26cm
\textwidth=17cm
\topmargin=-1.0cm
\oddsidemargin=0cm
\parindent=0mm

\input mycom.tex

\let\ds\displaystyle

\begin{document}

\pagenumbering{arabic}
\setcounter{page}{2} 

\vspace{-1.5cm}


\sf

{\bf I PARTE. A. Selección Única.} Cada una de las siguientes preguntas tiene una opción correcta, debe marcar con una equis dentro del paréntesis (X). (Total $20$ puntos, $1$ punto cada acierto).
%\vp

\benu
\item La intersección de dos planos diferentes es
\vp

\benu
\item[] \opc un plano.\vf
\item[] \opc un punto.\vf
\item[] \opc una recta.\vf
\item[] \opc un segmento.
\eenu
\vs

\item Dos o más rectas son coplanares si
\vp

\benu
\item[] \opc se intersecan en un punto.\vf
\item[] \opc determinan planos paralelos.\vf
\item[] \opc existe un plano que las contiene.\vf
\item[] \opc por ellas pasa infinita cantidad de planos.
\eenu
\vs

\item Tres o más puntos son colineales si
\vp

\benu
\item[] \opc determinan rectas paralelas.\vf
\item[] \opc existe un plano que los contiene.\vf
\item[] \opc existe una recta que los contiene.\vf
\item[] \opc por ellos pasan infinita cantidad de rectas.
\eenu
\vs

\item Si $\ell$ y $m$ son dos rectas coplanares que no se intersecan en ningún punto, entonces se de\-no\-mi\-nan
\vp

\benu
\item[] \opc paralelas.\vf
\item[] \opc alabeadas.\vf
\item[] \opc concurrentes.\vf
\item[] \opc perpendiculares.
\eenu

\pagebreak

\item Si dos ángulos forman un par lineal y uno de ellos mide $31\degre$, entonces el otro ángulo mide \vspace{2mm}
\vp

\benu
\item[] \opc $59\degre$\vf
\item[] \opc $31\degre$\vf
\item[] \opc $149\degre$\vf
\item[] \opc $180\degre$
\eenu
\vs

\item Si dos ángulos son congruentes y forman un par lineal, entonces cada uno mide \vp

\benu
\item[] \opc $45\degre$ \vf
\item[] \opc $90\degre$ \vf
\item[] \opc $180\degre$ \vf
\item[] \opc $360\degre$
\eenu\vs

\item El suplemento de un ángulo de medida $35\degre$, es un ángulo que mide \vp

\benu
\item[] \opc $35\degre$ \vf
\item[] \opc $55\degre$ \vf
\item[] \opc $145\degre$ \vf
\item[] \opc $180\degre$
\eenu\vs

\item Si un ángulo mide $10\degre$, la medida del suplemento de su complemento corresponde a \vp

\benu
\item[] \opc $10\degre$ \vf
\item[] \opc $80\degre$ \vf
\item[] \opc $100\degre$ \vf
\item[] \opc $180\degre$
\eenu

\pagebreak

\item De acuerdo con los datos de la figura tridimensional, dos rectas alabeadas corresponden a
\parbox{7.5cm}
{\vspace{-3cm}\benu
\item[] (  ) $\stackrel{\longleftrightarrow}{EH}$ y $\stackrel{\longleftrightarrow}{AD}$
\item[] (  ) $\stackrel{\longleftrightarrow}{FB}$ y $\stackrel{\longleftrightarrow}{DC}$
\item[] (  ) $\stackrel{\longleftrightarrow}{BC}$ y $\stackrel{\longleftrightarrow}{AD}$
\item[] (  ) $\stackrel{\longleftrightarrow}{AB}$ y $\stackrel{\longleftrightarrow}{FB}$
\eenu}
\parbox{8cm}
{\begin{tikzpicture}[line cap=round,line join=round,>=triangle 45,x=1.0cm,y=1.0cm, scale = .75]
\clip(-4.3,-5.44) rectangle (6,7);
\draw [line width=1pt] (-3.,0.)-- (-1.06,1.58);
\draw [line width=1pt] (-1.06,1.58)-- (3.56,1.6);
\draw [line width=1pt] (3.56,1.6)-- (2.,0.);
\draw [line width=1pt] (2.,0.)-- (-3.,0.);
\draw (-3.06,2.58)-- (-1.08,3.86);
\draw (-1.08,3.86)-- (3.6,3.9);
\draw (3.6,3.9)-- (2.,2.58);
\draw (2.,2.58)-- (-3.06,2.58);
\draw (-3.,0.)-- (-3.06,2.58);
\draw (-1.06,1.58)-- (-1.08,3.86);
\draw (2.,0.)-- (2.,2.58);
\draw (3.56,1.6)-- (3.6,3.9);
\draw [->](-3.06,2.58)-- (-4.18,2.58);
\draw [->](2.,2.58)-- (4.26,2.56);
\draw [->](2.,0.)-- (1.2,-0.88);
\draw [->](3.56,1.6)-- (4.14,2.16);
\draw [->](2.,0.)-- (3.,0.);
\draw [->](-3.,0.)-- (-4.,0.);
\draw [->](-3.,0.)-- (-3.86,-0.7);
\draw [->](-1.06,1.58)-- (-0.3,2.18);
\draw [->](-1.08,3.86)-- (-1.08,5.08);
\draw [->](-1.06,1.58)-- (-1.06,0.5);
\begin{scriptsize}
\draw [fill=black] (-3.,0.) circle (1.75pt);
\draw[color=black] (-3.3,0.42) node {\large $A$};
\draw [fill=black] (-1.06,1.58) circle (1.75pt);
\draw[color=black] (-1.38,1.84) node {\large $B$};
\draw [fill=black] (3.56,1.6) circle (1.75pt);
\draw[color=black] (3.22,1.96) node {\large $C$};
\draw [fill=black] (2.,0.) circle (1.75pt);
\draw[color=black] (1.66,0.38) node {\large $D$};
\draw [fill=black] (-3.06,2.58) circle (1.75pt);
\draw[color=black] (-3.1,3.) node {\large $E$};
\draw [fill=black] (-1.08,3.86) circle (1.75pt);
\draw[color=black] (-0.75,4.22) node {\large $F$};
\draw [fill=black] (3.6,3.9) circle (1.75pt);
\draw[color=black] (3.74,4.26) node {\large $G$};
\draw [fill=black] (2.,2.58) circle (1.75pt);
\draw[color=black] (1.96,3.) node {\large $H$};
\end{scriptsize}
\end{tikzpicture}
}

\vspace{-2.5cm}
\item De acuerdo con los datos de la figura, si $V$, $X$ y $Z$ son colineales y la medida del $\angle YXV=113\degre$. ¿Cuánto mide el $\angle YXZ$?

\parbox{6.5cm}
{\vspace{-2.5cm}
\benu
\item[] \opc $23\degre$ \vf
\item[] \opc $67\degre$ \vf
\item[] \opc $113\degre$ \vf
\item[] \opc $180\degre$
\eenu} 
\parbox{9cm}{
\begin{tikzpicture}[line cap=round,line join=round,>=triangle 45,x=1.0cm,y=1.0cm, scale=.75]
\clip(-4,-5) rectangle (8,4.25);
\draw [shift={(1.18,0.3)}] (0,0) -- (64.79887635452492:0.6) arc (64.79887635452492:180.:0.6) -- cycle;
\draw [shift={(1.18,0.3)}] (0,0) -- (0.:0.6) arc (0.:64.79887635452492:0.6) -- cycle;
\draw (-0.96,0.3)-- (3.46,0.3);
\draw [->](3.46,0.3)-- (4.54,0.3);
\draw [->](-0.96,0.3)-- (-1.92,0.3);
\draw (1.18,0.3)-- (2.14,2.34);
\draw [->](2.14,2.34)-- (2.54,3.18);
\begin{scriptsize}
\draw [fill=black] (-0.96,0.3) circle (1.75pt);
\draw[color=black] (-1.06,-0.1) node {\large $V$};
\draw [fill=black] (3.46,0.3) circle (1.75pt);
\draw[color=black] (3.46,-0.1) node {\large $Z$};
\draw [fill=black] (1.18,0.3) circle (1.75pt);
\draw[color=black] (1.12,-0.1) node {\large $X$};
\draw [fill=black] (2.14,2.34) circle (1.75pt);
\draw[color=black] (1.86,2.8) node {\large $Y$};
\end{scriptsize}
\end{tikzpicture}
}

\vspace{-2cm}
\item De acuerdo con los datos de la figura, $\angle DOB$ y $\angle AOC$ son 

\parbox{6cm}
{\vspace{-5cm}
\benu
\item[] \opc opuestos por el vértice. \vf
\item[] \opc complementarios. \vf
\item[] \opc suplementarios. \vf
\item[] \opc adyacentes.
\eenu} 
\parbox{6cm}
{\hspace{-3cm} \vf
\begin{tikzpicture}[line cap=round,line join=round,>=triangle 45,x=1.0cm,y=1.0cm]
\clip(-4,-5) rectangle (8,5);
\draw (2.,2.)-- (7.,2.);
\draw (4.5,2.)-- (4.5,4.4);
\draw (4.5,2.)-- (6.3,3.74);
\draw (4.5,2.)-- (3.26,0.8);
\draw (3.26,0.8)-- (2.72,0.28);
\draw [->](3.26,0.8)-- (2.72,0.28);
\draw [->](2.,2.)-- (1.42,2.);
\draw [->](7.,2.)-- (7.6,2.);
\draw [->](4.5,4.4)-- (4.5,5.);
\draw [->](6.3,3.74)-- (6.8,4.2);
\draw (4.14,2.)-- (4.5,2.);
\draw (4.5,2.)-- (4.5,2.36);
\draw (4.5,2.36)-- (4.14,2.36);
\draw (4.14,2.36)-- (4.14,2.);
\begin{scriptsize}
\draw [fill=black] (2.,2.) circle (1.0pt);
\draw[color=black] (1.96,2.38) node {\large $A$};
\draw [fill=black] (7.,2.) circle (1.0pt);
\draw[color=black] (6.98,2.34) node {\large $B$};
\draw [fill=black] (4.5,2.) circle (1.0pt);
\draw[color=black] (4.6,1.6) node {\large $O$};
\draw [fill=black] (4.5,4.4) circle (1.0pt);
\draw[color=black] (4.8,4.46) node {\large $E$};
\draw [fill=black] (6.3,3.74) circle (1.0pt);
\draw[color=black] (6.55,3.44) node {\large $D$};
\draw [fill=black] (3.26,0.8) circle (1.0pt);
\draw[color=black] (3.5,0.5) node {\large $C$};
\end{scriptsize}
\end{tikzpicture}
}

\pagebreak

\item El complemento del suplemento de un ángulo de medida $106\degre$, es un ángulo que mide \vp

\benu
\item[] \opc $16\degre$ \vf
\item[] \opc $74\degre$ \vf
\item[] \opc $86\degre$ \vf
\item[] \opc $90\degre$
\eenu\vs

\item  Considere las siguientes proposiciones. \\

\begin{tcolorbox}
\bc
\benu
\item[I.] Si $m\perp n$ entonces $m\parallel n$ \vp
\item[II.] Si $m\parallel n$ entonces $m\cap n=\,\,$\O
\eenu
\ec
\end{tcolorbox}

De ellas, ¿cuáles son {\sc verdaderas}?
\benu
\item[] \opc I.
\item[] \opc II.
\item[] \opc Ambas.
\item[] \opc Ninguna.
\eenu
\vs

\item Considere las siguientes afirmaciones. \vp

\begin{tcolorbox}
\benu
\item[I.] Dos ángulos son congruentes si suman $180\degre$.
\item[II.] Dos ángulos adyacentes siempre son complementarios.
\item[III.] Un ángulo agudo y uno obtuso siempre son suplementarios.
\item[IV.] La bisectriz de un ángulo llano lo divide en dos ángulos rectos.
\eenu
\end{tcolorbox}

¿Cuál de ellas es {\sc verdadera}?
\benu
\item[] \opc I
\item[] \opc II
\item[] \opc III
\item[] \opc IV
\eenu
\vs

\pagebreak

\item De acuerdo con la figura adjunta, es verdadero que

\parbox{6cm}
{\vspace{-1cm}
\benu
\item[] \opc $\stackrel{\longleftrightarrow}{PQ}\,\in\pi$ \vf
\item[] \opc $\stackrel{\longleftrightarrow}{PQ}\,\not\in\pi$ \vf
\item[] \opc $A\in\,\stackrel{\longleftrightarrow}{PQ}$ \vf
\item[] \opc $\stackrel{\longleftrightarrow}{AD}\,\parallel\,\stackrel{\longleftrightarrow}{PQ}$
\eenu} 
\parbox{3cm}
{\hspace{-3cm}\vspace{-2cm}
\begin{tikzpicture}[line cap=round,line join=round,>=triangle 45,x=1.0cm,y=1.0cm,scale=.65]
\clip(-4,-4) rectangle (8,6.3);
\draw (-1.88,1.28)-- (3.24,1.28);
\draw (3.24,1.28)-- (5.28,-1.46);
\draw (5.28,-1.46)-- (0,-1.46);
\draw (0,-1.46)-- (-1.88,1.28);
\draw [<->](-3.14,2.56)-- (4.1,2.56);
\draw (-2.1,2.56)-- (3.12,2.56);
\begin{scriptsize}
\draw [fill=black] (-1.88,1.28) circle (1.5pt);
\draw[color=black] (-2.36,1.44) node {\large $A$};
\draw [fill=black] (3.24,1.28) circle (1.5pt);
\draw[color=black] (3.7,1.42) node {\large $B$};
\draw [fill=black] (5.28,-1.46) circle (1.5pt);
\draw[color=black] (5.32,-2) node {\large $C$};
\draw [fill=black] (0,-1.46) circle (1.5pt);
\draw[color=black] (-0.08,-2) node {\large $D$};
\draw [fill=black] (-2.1,2.56) circle (1.5pt);
\draw[color=black] (-1.96,3.1) node {\large $P$};
\draw [fill=black] (3.12,2.56) circle (1.5pt);
\draw[color=black] (3.26,3.1) node {\large $Q$};
\draw[color=black] (-0.8,0.76) node {\Large $\pi$};
\end{scriptsize}
\end{tikzpicture}
}

\vspace{-0.3cm}
\item Si $\ell_1\parallel\ell_2$ y $\ell_2\perp\ell_3$, entonces, con certeza se cumple que \vs
\benu
\item[] \opc $\ell_1\parallel\ell_3$ \vf
\item[] \opc $\ell_1\perp\ell_3$ \vf
\item[] \opc $\ell_1\parallel\ell_2\parallel\ell_3$ \vf
\item[] \opc $\ell_1\cap\ell_3=\,\,$\O
\eenu
\vs

\item La expresión simbólica \,\,$\ell\cap\pi=\,\,$\O \, significa que la \vp

\benu
\item[] \opc recta $\ell$ pertenece al plano $\pi$.\vf
\item[] \opc recta $\ell$ es paralela al plano $\pi$. \vf
\item[] \opc recta $\ell$ es perpendicular al plano $\pi$. \vf
\item[] \opc intersección de la recta $\ell$ y el plano $\pi$ es un punto.
\eenu
\vs

\item La suma de las medidas de un ángulo recto y uno agudo genera un ángulo clasificado como \vs

\benu
\item[] \opc llano. \vf
\item[] \opc recto. \vf
\item[] \opc agudo. \vf
\item[] \opc obtuso.
\eenu

\pagebreak

\item De acuerdo con los datos de la figura, dos ángulos suplementarios son

\vspace{-1cm}
\parbox{5.5cm}
{\hspace{-2cm}\vspace{1.5cm}
\benu
\item[] \opc $\angle EBC$ y $\angle EBD$. \vf
\item[] \opc $\angle ABD$ y $\angle DBE$. \vf
\item[] \opc $\angle ABD$ y $\angle BEC$. \vf
\item[] \opc $\angle DBC$ y $\angle ABD$.
\eenu}
\parbox{8cm}
{\hspace{-3cm}\vspace{-3cm}
\begin{tikzpicture}[line cap=round,line join=round,>=triangle 45,x=1.0cm,y=1.0cm]
\clip(-3,-5) rectangle (10,4);
\draw (3.28,-1.48)-- (8.52,0.2);
\draw (5.9,-0.64)-- (3.16,0.92);
\draw [->](3.16,0.92)-- (2.56,1.28);
\draw (5.9,-0.64)-- (5.272,1.317);
\draw [->](5.272,1.317)-- (5.0485,2.0155);
\draw [->](8.52,0.2)-- (9.22,0.44);
\draw [->](3.28,-1.48)-- (2.68,-1.66);
\draw (5.782,-0.272)-- (5.9,-0.64);
\draw (5.9,-0.64)-- (6.267,-0.522);
\draw (6.267,-0.522)-- (6.15,-0.15);
\draw (6.15,-0.15)-- (5.782,-0.272);
\begin{scriptsize}
\draw [fill=black] (3.28,-1.48) circle (1.0pt);
\draw[color=black] (3.26,-1.84) node {\large $A$};
\draw [fill=black] (8.52,0.2) circle (1.0pt);
\draw[color=black] (8.56,-0.1) node {\large $C$};
\draw [fill=black] (5.272385733157199,1.3175587846763537) circle (1.0pt);
\draw[color=black] (5.5,1.58) node {\large $E$};
\draw [fill=black] (5.9,-0.64) circle (1.5pt);
\draw[color=black] (5.92,-1.02) node {\large $B$};
\draw [fill=black] (3.16,0.92) circle (1.0pt);
\draw[color=black] (2.98,0.62) node {\large $D$};
\end{scriptsize}
\end{tikzpicture}
}
\vs

\item  Considere las siguientes proposiciones. \vs

\begin{tcolorbox}
\benu
\item[I.] Los ángulos opuestos por el vértice son consecutivos.
\item[II.] El suplemento de un ángulo obtuso es un ángulo agudo.
\item[III.] El complemento de un ángulo agudo es un ángulo agudo.
\item[IV.] Si dos ángulos son suplementarios, uno de ellos debe ser obtuso.
\eenu
\end{tcolorbox}
\vp

De ellas, ¿cuáles son {\sc verdaderas}? \vp

\benu
\item[] \opc I y IV \vf
\item[] \opc II y III \vf
\item[] \opc I, II y III \vf
\item[] \opc II, III y IV
\eenu

\eenu

\pagebreak

{\bf B. Respuesta Corta.} Los siguientes ejercicios deben ser resueltos en forma concisa y breve. Sea ordenado y cuidadoso en sus respuestas. (Total 15 puntos, 1 punto cada respuesta correcta).\vp

\benu
\item De acuerdo con la figura adjunta, responda lo que se le solicita en cada espacio. \vs

\begin{multicols}{2}

\hspace{-1cm}
{\benu
\item[{\boldmath $a)$}] Un par lineal \compl.
\vp
\item[{\boldmath $b)$}] Un ángulo recto \compl.
\vp
\item[{\boldmath $c)$}] Un ángulo agudo \compl.
\vp
\item[{\boldmath $d)$}] Un ángulo obtuso \compl.
\vp
\item[{\boldmath $e)$}] Tres puntos colineales \compl.
\eenu
}

{\hspace{8mm}}
\vs\vs

\begin{tikzpicture}[line cap=round,line join=round,>=triangle 45,x=1.0cm,y=1.0cm, scale=.75]
\clip(-4.3,-5.44) rectangle (15.98,6.3);
\draw (2.54,4.08)-- (6.16,4.08);
\draw (6.16,4.08)-- (6.16,-2.52);
\draw (6.16,-2.52)-- (-2.22,-2.52);
\draw (-2.22,-2.52)-- (2.54,4.08);
\draw (6.16,-2.52)-- (2.54,4.08);
\draw (6.16,4.08)-- (-2.22,-2.52);
\draw [line width=0.4pt](6.16,4.08)-- (5.72,4.08);
\draw [line width=0.4pt](5.72,4.08)-- (5.72,3.64);
\draw [line width=0.4pt](5.72,3.64)-- (6.16,3.64);
\draw [line width=0.4pt](6.16,3.64)-- (6.16,4.08);
\draw [line width=0.4pt](6.16,-2.08)-- (5.72,-2.08);
\draw [line width=0.4pt](5.72,-2.08)-- (5.72,-2.52);
\draw [line width=0.4pt] (5.72,-2.52)-- (6.16,-2.52);
\draw [line width=0.4pt] (6.16,-2.52)-- (6.16,-2.08);
\begin{scriptsize}
\draw [fill=black] (2.54,4.08) circle (1.75pt);
\draw[color=black] (2.26,4.5) node {\large $A$};
\draw [fill=black] (6.16,4.08) circle (1.75pt);
\draw[color=black] (6.36,4.55) node {\large $B$};
\draw [fill=black] (6.16,-2.52) circle (1.75pt);
\draw[color=black] (6.3,-2.95) node {\large $C$};
\draw [fill=black] (-2.22,-2.52) circle (1.75pt);
\draw[color=black] (-2.46,-2.95) node {\large $D$};
\draw [fill=black] (3.63,2.089) circle (1.75pt);
\draw[color=black] (3.45,1.45) node {\large $E$};
\end{scriptsize}
\end{tikzpicture}

\end{multicols}

\vspace{-1cm}

\item De acuerdo con los datos de la figura tridimensional la cual representa un prisma de base hexa\-go\-nal, responda lo que se le solicita en el espacio indicado.\vs

\hspace{-1cm}
\begin{multicols}{2}

\benu
\item[{\boldmath $a)$}] Número de caras \compl.
\vs
\item[{\boldmath $b)$}] Número de aristas \compl.
\vs
\item[{\boldmath $c)$}] Número de vértices \compl.
\vs
\item[{\boldmath $d)$}] Dos segmentos paralelos \compl.
\vspace{1mm}
\item[{\boldmath $e)$}] Dos planos perpendiculares \compl.
\eenu

\vspace{-1cm}
\vs\vs

\begin{tikzpicture}[line cap=round,line join=round,>=triangle 45,x=1.0cm,y=1.0cm]
\clip(-2,-2) rectangle (5.3,5);
\draw [dash pattern=on 2pt off 2pt] (0.,0.)-- (1.88,0.82);
\draw [dash pattern=on 2pt off 2pt] (1.88,0.82)-- (4.22,0.54);
\draw [dash pattern=on 2pt off 2pt] (4.22,0.54)-- (4.86,-0.46);
\draw (4.86,-0.46)-- (3.22,-1.32);
\draw (3.22,-1.32)-- (0.68,-0.94);
\draw (0.68,-0.94)-- (0.,0.);
\draw (1.84,3.76)-- (4.18,3.48);
\draw (4.18,3.48)-- (4.82,2.48);
\draw (4.82,2.48)-- (3.18,1.62);
\draw (3.18,1.62)-- (0.64,2.);
\draw (0.,3.)-- (1.84,3.76);
\draw (1.84,3.76)-- (4.18,3.48);
\draw (4.18,3.48)-- (4.82,2.48);
\draw (4.82,2.48)-- (3.18,1.62);
\draw (3.18,1.62)-- (0.64,2.);
\draw (0.64,2.)-- (0.,3.);
\draw [dash pattern=on 2pt off 2pt] (1.88,0.82)-- (1.84,3.76);
\draw (3.22,-1.32)-- (3.18,1.62);
\draw [dash pattern=on 2pt off 2pt] (4.22,0.54)-- (4.18,3.48);
\draw (4.86,-0.46)-- (4.82,2.48);
\draw (0.68,-0.94)-- (0.64,2.);
\draw (0.,0.)-- (0.,3.);
\draw (0.,0.)-- (0.,3.);
\begin{scriptsize}
\draw [fill=black] (0.,0.) circle (1.0pt);
\draw[color=black] (-0.28,0) node {\large $A$};
\draw [fill=black] (1.88,0.82) circle (1.0pt);
\draw[color=black] (1.5,1.06) node {\large $B$};
\draw [fill=black] (4.22,0.54) circle (1.0pt);
\draw[color=black] (4.5,0.74) node {\large $C$};
\draw [fill=black] (4.86,-0.46) circle (1.0pt);
\draw[color=black] (5.1,-0.52) node {\large $D$};
\draw [fill=black] (3.22,-1.32) circle (1.0pt);
\draw[color=black] (3.32,-1.6) node {\large $E$};
\draw [fill=black] (0.68,-0.94) circle (1.0pt);
\draw[color=black] (0.52,-1.3) node {\large $F$};
\draw [fill=black] (1.84,3.76) circle (1.0pt);
\draw[color=black] (1.8,4.08) node {\large $H$};
\draw [fill=black] (4.18,3.48) circle (1.0pt);
\draw[color=black] (4.32,3.8) node {\large $I$};
\draw [fill=black] (4.82,2.48) circle (1.0pt);
\draw[color=black] (5.1,2.72) node {\large $J$};
\draw [fill=black] (3.18,1.62) circle (1.0pt);
\draw[color=black] (3.16,1.96) node {\large $K$};
\draw [fill=black] (0.64,2.) circle (1.0pt);
\draw[color=black] (0.74,2.38) node {\large $L$};
\draw [fill=black] (0.,3.) circle (1.0pt);
\draw[color=black] (-0.26,3.16) node {\large $G$};
\end{scriptsize}
\end{tikzpicture}

\end{multicols}

\pagebreak

\hspace{-1cm}
\item Complete los espacios en blanco con lo solicitado en cada caso. \vs

\hspace{-1cm}
{\boldmath $a)$} Dos rectas secantes determinan 2 pares de ángulos \compl. \vs

\hspace{-1cm}
{\boldmath $b)$} Un par de ángulos que poseen un lado y un vértice en común se denominan \compl.\vs

\hspace{-1cm}
{\boldmath $c)$} Un rayo que corta un ángulo y determina en él dos ángulos congruentes se llama \compl.\vs

\hspace{-1cm}
{\boldmath $d)$} Si dos ángulos son congruentes y complementarios, entonces cada un o mide \compl.\vs

\hspace{-1cm}
{\boldmath $e)$} Si una bisectriz corta a un ángulo obtuso, entonces cada uno de los ángulos que se obtienen, según su medida, se clasifican como \compl.

\vs\vs
\eenu

{\bf II PARTE. Desarrollo.} Resuelva los siguientes problemas con orden y claridad. Deben aparecer todos los procedimientos que justifican la respuesta, en el espacio indicado. (Total 15 puntos). \vp

\benu
\item Según los datos de la figura, las rectas $\stackrel{\leftarrow\!\!\!\lra}{GC}$ y $\stackrel{\leftarrow\!\!\!\lra}{FD}$ se intersecan en el punto $E$. Si $\stackrel{\lra}{EB}$ es la bisectriz del $\angle FEC$, ¿cuál es la medida del $\,\angle BED$? \hfill{\bf (4 puntos)} \vs

\begin{tikzpicture}[line cap=round, line join=round, >=triangle 45,x=1.0cm,y=1.0cm]
\clip(-3,-4) rectangle (6,5);
\draw [shift={(1.44,1.48)},line width=0.4pt] (0,0) -- (180.:1.5) arc (180.:207.05755291084134:1.5) -- cycle;
\draw [shift={(1.44,1.48)},line width=0.4pt] (0,0) -- (0.:1.3) arc (0.:110.69545073406329:1.3) -- cycle;
\draw [<->](-2.56,1.48) -- (5.38,1.48);
\draw [<->](-2.152364337903292,-0.3549486905836958)-- (5.103912650849684,3.3514949511534358);
\draw [<-](0.15287813310285228,4.887087294727744)-- (1.4326534140017286,1.499446845289542);
\begin{scriptsize}
\draw [fill=black] (1.44,1.48) circle (1.5pt);
\draw[color=black] (1.56,1) node {\large $E$};
\draw [fill=black] (4.22,2.9) circle (1.5pt);
\draw[color=black] (4.12,3.3) node {\large $C$};
\draw [fill=black] (0.42,4.18) circle (1.5pt);
\draw[color=black] (0.65,4.46) node {\large $B$};
\draw [fill=black] (-1.46,1.48) circle (1.5pt);
\draw[color=black] (-1.38,1.9) node {\large $F$};
\draw [fill=black] (4.44,1.48) circle (1.5pt);
\draw[color=black] (4.42,1.16) node {\large $D$};
\draw [fill=black] (-1.1015839146491584,0.18178087812884702) circle (1.5pt);
\draw[color=black] (-0.86,-0.02) node {\large $G$};
\draw[color=black] (-0.55,1.1) node {\large $28\textrm{\degre}$};
\end{scriptsize}
\end{tikzpicture}

\pagebreak

\item Luis repara carretas típicas y recibió una rueda a la que le hacen falta algunos rayos. Antes de sustituir los restantes, debe medir los ángulos que forman los rayos de la rueda. Encuentre las medidas de los ángulos $\angle HAE$,$\angle FAD$ y $\angle BAG$, si se sabe que $D-A-E$ y $F-A-G$. {\bf (5 puntos)}

\vspace{-1.3cm}

\begin{tikzpicture}[line cap=round,line join=round,>=triangle 45,x=1.0cm,y=1.0cm]
\clip(-4.3,-5.14) rectangle (10.02,6.3);
\draw [shift={(0.,0.)}] (0,0) -- (-10.738897100905442:1.) arc (-10.738897100905442:45.22646435234117:1.) -- cycle;
\draw [shift={(0.,0.)}] (0,0) -- (-134.53794727856925:1.) arc (-134.53794727856925:-102.45824644000491:1.) -- cycle;
\draw [shift={(0.,0.)}] (0,0) -- (115.49279547280312:1.) arc (115.49279547280312:169.8409440041532:1.) -- cycle;
\draw(0.,0.) circle (4.cm);
\draw(0.,0.) circle (3.52cm);
\draw (-2.46886287945264,-2.509006991313659)-- (2.4791584967594087,2.498834357844799);
\draw (-3.4648130977767173,0.6208624626073879)-- (3.458352565560486,-0.6558944520890577);
\draw (0.,0.)-- (-1.5149995515630956,3.177290726194822);
\draw (0.,0.)-- (-0.7593628888127346,-3.437116233573427);
\begin{scriptsize}
\draw [fill=black] (0.,0.) circle (1.0pt);
\draw[color=black] (0.14,-0.34) node {\large $A$};
\draw [fill=black] (-2.469,-2.509) circle (1.0pt);
\draw[color=black] (-2.5,-2) node {\large $D$};
\draw [fill=black] (2.479,2.498) circle (1.0pt);
\draw[color=black] (2.43,2.06) node {\large $E$};
\draw [fill=black] (-3.464,0.621) circle (1.0pt);
\draw[color=black] (-3.1,0.94) node {\large $F$};
\draw [fill=black] (3.458,-0.655) circle (1.0pt);
\draw[color=black] (3,-1) node {\large $G$};
\draw [fill=black] (-1.515,3.177) circle (1.0pt);
\draw[color=black] (-1.,3.) node {\large $H$};
\draw [fill=black] (-0.759,-3.437) circle (1.0pt);
\draw[color=black] (-0.35,-3.16) node {\large $B$};
\draw[color=black] (1.55,0.4) node {\large $60\textrm{\degre}$};
\draw[color=black] (-0.71,-1.38) node {\large $32\textrm{\degre}$};
\draw[color=black] (-1.2,0.68) node {\large $54\textrm{\degre}$};
\end{scriptsize}
\end{tikzpicture}
\vs


\item Dos ángulos son suplementarios y se sabe que uno mide el doble de la medida del
otro. Encuentre las medidas de los dos ángulos. \hfill{\bf (3 puntos)}

\vs\vs\vs\vs\vs\vs\vs\vs

\item ¿Cuál es el suplemento de un ángulo que es complemento de otro que mide $45\degre$? {\bf (3 puntos)}
\eenu


\end{document}