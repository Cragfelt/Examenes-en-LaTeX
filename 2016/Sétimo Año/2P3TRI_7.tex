\documentclass[12pt, fleqn]{article}
\usepackage[left=1in, right=1in, top=1in, bottom=1in]{geometry}
\usepackage{mathexam}
\usepackage{amsmath}
\usepackage{multicol}
\usepackage{textcomp}
\usepackage[utf8x]{inputenc}
\usepackage{pgfplots}
\usepackage{wasysym}
\usepackage{graphicx}
\usepackage{tcolorbox}
\usepackage{pdfpages}

\usepackage{pstricks-add}
\usepackage{pst-solides3d}

\pgfplotsset{width=7cm,compat=1.11}
\usepackage{pgf,tikz}
\usepackage{mathrsfs}
\usetikzlibrary{arrows}

\pagestyle{myheadings}

\textheight=26cm
\textwidth=17cm
\topmargin=-1.0cm
\oddsidemargin=0cm
\parindent=0mm

\input mycom.tex

\let\ds\displaystyle

\begin{document}

\pagenumbering{arabic}
\setcounter{page}{2}

\vspace{-1.5cm}


\sf

{\bf I PARTE. A. Selección Única.} Cada una de las siguientes preguntas tiene una opción correcta, debe marcar con una equis dentro del paréntesis (X). (Total $20$ puntos, $1$ punto cada acierto).
%\vp

\benu %SELECCIÓN ÚNICA
\item Si dos de los ángulos internos de un triángulo miden $90\degre$ y $32\degre$, entonces la medida del tercer ángulo interno es \vp

\benu
\item[] \opc $58\degre$\vf
\item[] \opc $62\degre$\vf
\item[] \opc $122\degre$\vf
\item[] \opc $180\degre$
\eenu
\vs

\item ¿Cuál terna de números corresponde a las medidas de los lados de un triángulo?
\vp

\benu
\item[] \opc $(5,1,2)$\vf
\item[] \opc $(1,3,5)$\vf
\item[] \opc $(9,3,8)$\vf
\item[] \opc $(2,7,10)$
\eenu
\vs

\item Dos de los lados de un triángulo miden $8$ y $12$, entonces la medida del tercer lado puede ser
\vp

\benu
\item[] \opc $2$\vf
\item[] \opc $3$\vf
\item[] \opc $4$\vf
\item[] \opc $5$
\eenu
\vs

\item En un triángulo equilátero, la medida de un ángulo externo es de
\vp

\benu
\item[] \opc $45\degre$\vf
\item[] \opc $60\degre$\vf
\item[] \opc $90\degre$\vf
\item[] \opc $120\degre$
\eenu

\pagebreak

\item Si en un triángulo dos de las medidas de los ángulos internos son de $\,40\degre\,$ cada uno, entonces la medida del tercer ángulo interno es de \vp

\benu
\item[] \opc $45\degre$\vf
\item[] \opc $80\degre$\vf
\item[] \opc $100\degre$\vf
\item[] \opc $180\degre$
\eenu
\vs

\item Un triángulo que posee un ángulo interno obtuso se denomina \vp

\benu
\item[] \opc rectángulo. \vf
\item[] \opc equiángulo. \vf
\item[] \opc acutángulo. \vf
\item[] \opc obtusángulo.
\eenu
\vs

\item La suma de las medidas de los ángulos externos de todo triángulo es \vp

\benu
\item[] \opc $60\degre$ \vf
\item[] \opc $90\degre$ \vf
\item[] \opc $180\degre$ \vf
\item[] \opc $360\degre$
\eenu\vs

\item Un triángulo isósceles es aquel que posee \vp

\benu
\item[] \opc tres lados congruentes. \vf
\item[] \opc dos ángulos congruentes. \vf
\item[] \opc tres ángulos de igual medida. \vf
\item[] \opc dos lados de diferente medida.
\eenu

\pagebreak

\item De acuerdo con los datos de la figura, ¿cuál es la medida del $\angle\,\alpha$?
\vspace{-2mm}

  \begin{minipage}{\linewidth}
      \begin{minipage}{0.45\linewidth}
          \benu
            \item[] \opc $35\degre$ \vf
            \item[] \opc $45\degre$ \vf
            \item[] \opc $65\degre$ \vf
            \item[] \opc $115\degre$
            \eenu
      \end{minipage}
      \hspace{-1.5cm}
      \begin{minipage}{0.45\linewidth}
         \begin{tikzpicture}[line cap=round,line join=round,>=triangle 45,x=1.0cm,y=1.0cm, scale=0.75]
            \clip(-4,-5) rectangle (8,4);
            \draw [shift={(1.124,1.725)}] (0,0) -- (-46.769:0.6) arc (-46.769:53.923:0.6) -- cycle;
            \draw [shift={(-0.536,-0.553)}] (0,0) -- (-11.160:0.6) arc (-11.160:53.923:0.6) -- cycle;
            \draw [shift={(4.133,-1.475)}] (0,0) -- (-46.769:0.6) arc (-46.769:-11.160:0.6) -- cycle;
            \draw [<->](-1.24,-1.52)-- (1.82,2.68);
            \draw [<->](0.34,2.56)-- (5.36,-2.78);
            \draw [<->](-1.62,-0.34)-- (5.78,-1.8);
            \begin{scriptsize}
            \draw[color=black] (2.5,1.8) node {\normalsize $100\degre$};
            \draw[color=black] (0.3,-0.14) node {\normalsize $\alpha$};
            \draw[color=black] (5.45,-2.1) node {\normalsize $35\degre$};
            \end{scriptsize}
        \end{tikzpicture}
      \end{minipage}
  \end{minipage}
\vspace{-1cm}

\item De acuerdo con los datos de la figura, ¿cuánto mide el $\angle\,\theta$?

\vspace{-3cm}
\begin{minipage}{\linewidth}
      \begin{minipage}{0.45\linewidth}
        \benu
            \item[] \opc $32\degre$ \vf
            \item[] \opc $64\degre$ \vf
            \item[] \opc $84\degre$ \vf
            \item[] \opc $96\degre$
        \eenu
      \end{minipage}
      \hspace{-1.5cm}
      \begin{minipage}{0.45\linewidth}
         \begin{tikzpicture}[line cap=round,line join=round,>=triangle 45,x=1.0cm,y=1.0cm]
            \clip(-4.3,-5.04) rectangle (6,6);
            \draw [shift={(0.74,-1.64)},line width=0.4pt] (0,0) -- (0.:0.6) arc (0.:105.7:0.6) -- cycle;
            \draw [shift={(-0.48,2.7)}] (0,0) -- (-115.188:0.6) arc (-115.1883:-74.3:0.6) -- cycle;
            \draw [shift={(-2.54,-1.68)}] (0,0) -- (0.698:0.6) arc (0.698:64.811:0.6) -- cycle;
            \draw [line width=0.4pt] (-2.54,-1.68)-- (-0.48,2.7);
            \draw [line width=0.4pt] (-0.48,2.7)-- (0.74,-1.64);
            \draw [line width=0.4pt] (0.74,-1.64)-- (-2.54,-1.68);
            \draw [dash pattern=on 3pt off 3pt] (0.74,-1.64)-- (2.4,-1.64);
            \begin{scriptsize}
            \draw[color=black] (1.4,-1) node {\normalsize $\theta$};
            \draw[color=black] (-0.62,1.5) node {\normalsize $32\textrm{\degre}$};
            \draw[color=black] (-1.6,-1.16) node {\normalsize $64\textrm{\degre}$};
            \end{scriptsize}
        \end{tikzpicture}
      \end{minipage}
\end{minipage}
\vspace{-3cm}

\item Considere las siguientes afirmaciones. \vp

\begin{tcolorbox}
\benu
\item[I.] Todo triángulo equilátero es acutángulo.
\item[II.] Un triángulo isósceles puede ser obtusángulo.
\item[III.] Un triángulo rectángulo puede ser también isósceles.
\item[IV.] Un triángulo puede tener dos ángulos externos agudos.
\eenu
\end{tcolorbox}

De ellas, ¿cuáles son {\sc verdaderas}?
\benu
\item[] \opc I y II
\item[] \opc II y III
\item[] \opc I, II y III
\item[] \opc I, III y IV
\eenu
\vs

\pagebreak

\item Dados tres segmentos de longitudes $40\,cm, 25\,cm$ y $30\,cm$, con certeza podemos afirmar que se forma un triángulo \vp

\benu
\item[] \opc escaleno. \vf
\item[] \opc isósceles. \vf
\item[] \opc equilátero.\vf
\item[] \opc rectángulo.
\eenu\vp

\item  Considere las siguientes proposiciones. \vp

\begin{tcolorbox}
\bc
\benu
\item[I.] Un triángulo puede tener tres ángulos rectos.
\item[II.] Un triángulo puede tener tres ángulos agudos.
\item[III.] Un triángulo puede tener tres ángulos obtusos.
\eenu
\ec
\end{tcolorbox}

De ellas, ¿cuáles son siempre {\sc verdaderas}?
\benu
\item[] \opc Todas.
\item[] \opc Solo I.
\item[] \opc Solo I y II.
\item[] \opc Solo II y III.
\eenu
\vp

\item  Considere las siguientes proposiciones. \vp

\begin{tcolorbox}
\benu
\item[I.] Un triángulo escaleno puede ser acutángulo.
\item[II.] Un triángulo isósceles puede ser acutángulo.
\item[III.] Todo triángulo equilátero es también isósceles.
\item[IV.] Todo triángulo rectángulo tiene dos ángulos internos agudos.
\eenu
\end{tcolorbox}
\vp

De ellas, ¿cuáles son {\sc verdaderas}? \vp

\benu
\item[] \opc Todas. \vf
\item[] \opc Solo I y II. \vf
\item[] \opc Solo I y II. \vf
\item[] \opc Solo III y IV. \vf
\eenu

\pagebreak

\item De acuerdo con la terna $(16,x,10)$, ¿cuál puede ser un valor de $x$ para que se forme un triángulo con esas medidas? \vp

\benu
\item[] \opc $5$\vf
\item[] \opc $6$\vf
\item[] \opc $10$\vf
\item[] \opc $26$
\eenu
\vs

\item Para que exista un triángulo, en la terna $(6,12,x)$ un posible valor para $x$ es \vp

\benu
\item[] \opc $5$\vf
\item[] \opc $6$ \vf
\item[] \opc $8$ \vf
\item[] \opc $18$
\eenu
\vs

\item Si en un triángulo sus tres ángulos internos son congruentes, el triángulo se denomina \vp

\benu
\item[] \opc escaleno. \vf
\item[] \opc isósceles. \vf
\item[] \opc equilátero. \vf
\item[] \opc obtusángulo.
\eenu
\vs

\item Si $71\degre$ y $124\degre$ son medidas de ángulos externos de un triángulo, el otro ángulo externo mide \vp

\benu
\item[] \opc $15\degre$\vf
\item[] \opc $165\degre$ \vf
\item[] \opc $195\degre$ \vf
\item[] \opc $360\degre$
\eenu
\vs

\pagebreak

\item Si $65\degre$ y $27\degre$ son medidas de ángulos internos de un triángulo, el otro ángulo interno mide \vp

\benu
\item[] \opc $88\degre$\vf
\item[] \opc $92\degre$ \vf
\item[] \opc $180\degre$ \vf
\item[] \opc $360\degre$
\eenu
\vs


\item Una terna que representa las medidas de los lados de un triángulo isósceles es \vp

\benu
\item[] \opc $(3,4,5)$\vf
\item[] \opc $(6,8,6)$ \vf
\item[] \opc $(6,6,16)$ \vf
\item[] \opc $(10,10,20)$
\eenu
\vs

\eenu

%\pagebreak

{\bf B. Respuesta Corta.} Escriba sobre el espacio en blanco la respuesta correcta según sea el caso. Sea ordenado y cuidadoso. (Total 15 puntos, 1 punto cada respuesta correcta).\vs

\benu
\item De acuerdo con los datos de la figura adjunta, si $\ell_1\parallel\ell_2$, determine y escriba la medida de cada uno de los ángulos indicados. \hfill{\bf (5 puntos)} \vs\vs\vs

\begin{minipage}{\linewidth}
\vspace{-.5cm}
      \begin{minipage}{0.45\linewidth}
        \benu
            \item[] $m\,\angle\,\alpha=$ \compl.\vp
            \item[] $m\,\angle\,\lambda=$ \compl.\vp
            \item[] $m\,\angle\,\beta=$ \compl.\vp
            \item[] $m\,\angle\,\theta=$ \compl.\vp
            \item[] $m\,\angle\,\delta=$ \compl.
        \eenu
      \end{minipage}
      \hspace{1cm}
      \begin{minipage}{0.45\linewidth}
         \begin{tikzpicture}[line cap=round,line join=round,>=triangle 45,x=1.0cm,y=1.0cm]
            \clip(-1.02,-0.82) rectangle (6.1,5.36);
            \draw [<->](-0.6,3.16)-- (5.5,3.16);
            \draw [<->](-0.6,1.54)-- (5.5,1.54);
            \draw [<->](0.2,-0.58)-- (4.28,5.02);
        \begin{scriptsize}
            \draw[color=black] (5.28,3.6) node {\ns $\ell_1$};
            \draw[color=black] (5.28,2) node {\ns $\ell_2$};
            \draw[color=black] (3.8,5.) node {\ns $\ell_3$};
            \draw[color=black] (3.54,3.5) node {\ns $\lambda$};
            \draw[color=black] (2.24,2.78) node {\ns $\alpha$};
            \draw[color=black] (3.29,2.78) node {\ns $104\degre$};
            \draw[color=black] (1.55,1.85) node {\ns $\beta$};
            \draw[color=black] (2.35,1.85) node {\ns $\delta$};
            \draw[color=black] (1.2,1.2) node {\ns $\theta$};
        \end{scriptsize}
        \end{tikzpicture}
      \end{minipage}
\end{minipage}

\vs\vs
\pagebreak

\item De acuerdo con los datos de la siguiente figura, determine y escriba el nombre que recibe cada una de las parejas de ángulos. \hfill{\bf (5 puntos)} \vs\vs\vs

\begin{minipage}{\linewidth}
      \begin{minipage}{0.45\linewidth}
            \begin{tikzpicture}[line cap=round,line join=round,>=triangle 45,x=1.0cm,y=1.0cm]
            \clip(-1.02,-0.82) rectangle (6.1,5.36);
            \draw [<->](-0.6,3.16)-- (5.5,3.16);
            \draw [<->](-0.6,1.54)-- (5.5,1.54);
            \draw [<->](4.28,-0.58)-- (0.2,5.02);
            \begin{scriptsize}
            \draw[color=black] (5.28,3.6) node {\ns $\ell_1$};
            \draw[color=black] (5.28,2) node {\ns $\ell_2$};
            \draw[color=black] (0.85,5.) node {\ns $\ell_3$};
            
            \draw[color=black] (1.75,3.5) node {\ns $\beta$};
            \draw[color=black] (1,3.5) node {\ns $\alpha$};
            \draw[color=black] (1.35,2.85) node {\ns $\omega$};
            \draw[color=black] (2.27,2.85) node {\ns $\delta$};
            \draw[color=black] (2.15,1.85) node {\ns $\varepsilon$};
            \draw[color=black] (2.9,1.85) node {\ns $\pi$};
            \draw[color=black] (2.45,1.2) node {\ns $\theta$};
            \draw[color=black] (3.4,1.2) node {\ns $\lambda$};
            \end{scriptsize}
        \end{tikzpicture}
      \end{minipage}
      %\hspace{.5cm}      
      \begin{minipage}{0.45\linewidth}
        \benu
            \item[] $\angle\,\beta$ y $\angle\,\theta$: \hrulefill.\vs
            \item[] $\angle\,\delta$ y $\angle\,\pi$: \hrulefill.\vs
            \item[] $\angle\,\varepsilon$ y $\angle\,\alpha$: \hrulefill.\vs
            \item[] $\angle\,\beta$ y $\angle\,\theta$: \hrulefill.\vs
            \item[] $\angle\,\pi$ y $\angle\,\omega$: \hrulefill.
        \eenu
      \end{minipage}
\end{minipage}

\vs\vs\vs

\item De acuerdo con los datos de la figura adjunta, determine y escriba la medida de cada uno de los ángulos indicados. \hfill{\bf (5 puntos)}
\vs\vs

\begin{minipage}{\linewidth}
      \begin{minipage}{0.45\linewidth}
      \vspace{-1cm}
        \benu
            \item[] $m\,\angle\,\alpha=$ \compl.\vp
            \item[] $m\,\angle\,\omega=$ \compl.\vp
            \item[] $m\,\angle\,\beta=$ \compl.\vp
            \item[] $m\,\angle\,\varepsilon=$ \compl.\vp
            \item[] $m\,\angle\,\theta=$ \compl.
        \eenu
      \end{minipage}
      \hspace{-.75cm}
      \begin{minipage}{0.45\linewidth}
        \begin{tikzpicture}[line cap=round,line join=round,>=triangle 45,x=1.0cm,y=1.0cm]
        \clip(1.52,-3.52) rectangle (10.5,5.56);
        \draw [shift={(5.44,4.)}] (0,0) -- (5.389:0.6) arc (5.389:86.248:0.6) -- cycle;
        \draw [shift={(8.62,4.3)}] (0,0) -- (5.262:0.6) arc (5.262:37.333:0.6) -- cycle;
        \draw [shift={(8.62,4.3)}] (0,0) -- (37.333:0.6) arc (37.3331957381858:185.38931175997342:0.6) -- cycle;
        \draw [shift={(5.241,1.466)}] (0,0) -- (39.979:0.6) arc (39.979:85.508:0.6) -- cycle;
        \draw [shift={(4.92,-2.62)}] (0,0) -- (-44.215:0.6) arc (-44.215:85.508:0.6) -- cycle;
        \draw [shift={(2.8,-0.58)}] (0,0) -- (39.979:0.6) arc (39.979:137.121:0.6) -- cycle;
        \draw [shift={(2.8,-0.58)}] (0,0) -- (137.121:0.6) arc (137.121:222.336:0.6) -- cycle;
        \draw [shift={(4.92,-2.62)}] (0,0) -- (85.508:0.6) arc (85.508:136.101:0.6) -- cycle;
        \draw (2.8,-0.58)-- (8.62,4.3);
        \draw (5.44,4.)-- (4.92,-2.62);
        \draw (4.92,-2.62)-- (2.8,-0.58);
        \draw (5.44,4.)-- (8.62,4.3);
        \draw [->](8.62,4.3)-- (10.14,4.44);
        \draw [->](8.62,4.3)-- (9.8,5.2);
        \draw [->](5.44,4.)-- (5.52,5.22);
        \draw [->](4.92,-2.62)-- (5.66,-3.34);
        \draw [->](2.8,-0.58)-- (1.96,0.2);
        \draw [->](2.8,-0.58)-- (1.9,-1.4);
        \begin{scriptsize}
        \draw[color=black] (6.3,4.7) node {\ns $80\textrm{\degre}$};
        \draw[color=black] (9.74,4.7) node {\ns $\beta$};
        \draw[color=black] (8.64,5.2) node {\ns $\theta$};
        \draw[color=black] (5.8,2.32) node {\ns $45\textrm{\degre}$};
        \draw[color=black] (5.8,-2.28) node {\ns $\varepsilon$};
        \draw[color=black] (2.8,0.34) node {\ns $\omega$};
        \draw[color=black] (1.85,-0.5) node {\ns $\alpha$};
        \draw[color=black] (4.6,-1.66) node {\ns $50\textrm{\degre}$};
        \end{scriptsize}
    \end{tikzpicture}
    \end{minipage}
\end{minipage}

\pagebreak

\item De acuerdo con el siguiente plano cartesiano, escriba las coordenadas de los puntos que se le solicitan. \hfill{\bf (5 puntos)}\vs\vs

\begin{minipage}{\linewidth}
      \begin{minipage}{0.5\linewidth}
      
        {$A:$ \compl.\vs
        
        $B:$ \compl.\vs
        
        $C:$ \compl.\vs
        
        $D:$ \compl.\vs
        
        %$E:$ \compl.\vs
        
        Punto medio de $\overline{AB}$ \compl.}
      \end{minipage}
      \hspace{-.5cm}
      \begin{minipage}{0.75\linewidth}
\definecolor{cqcqcq}{rgb}{0.7529411764705882,0.7529411764705882,0.7529411764705882}
\begin{tikzpicture}[line cap=round,line join=round,>=triangle 45,x=1.0cm,y=1.0cm,scale=.65]
\draw [color=cqcqcq,dotted, xstep=1.0cm,ystep=1.0cm] (-6.84,-5.38) grid (5.98,5.96);
\draw[<->,color=black] (-6,0.) -- (5.98,0.);
\foreach \x in {-5,-4,-3,-2,-1,1,2,3,4,5}
\draw[shift={(\x,0)},color=black] (0pt,2pt) -- (0pt,-2pt) node[below] {\footnotesize $\x$};
\draw[<->,color=black] (0.,-5.38) -- (0.,5.96);
\foreach \y in {-5,-4,-3,-2,-1,1,2,3,4,5}
\draw[shift={(0,\y)},color=black] (2pt,0pt) -- (-2pt,0pt) node[left] {\footnotesize $\y$};
\draw[color=black] (0pt,-10pt) node[right] {\footnotesize $0$};
\clip(-6.84,-5.38) rectangle (5.98,5.96);
\begin{scriptsize}
\draw [fill=black] (-3.,2.) circle (1.75pt);
\draw[color=black] (-3,2.6) node {\ns $A$};
\draw [fill=black] (1.,-4.) circle (1.75pt);
\draw[color=black] (1,-3.4) node {\ns $B$};
\draw [fill=black] (-5.,0.) circle (1.75pt);
\draw[color=black] (-5,0.6) node {\ns $C$};
\draw [fill=black] (2.,5.) circle (1.75pt);
\draw[color=black] (2.5,5) node {\ns $D$};
%\draw [fill=black] (-4.,-3.) circle (1.75pt);
%\draw[color=black] (-4.,-3.55) node {\ns $E$};
\end{scriptsize}
\end{tikzpicture}
\end{minipage}
\end{minipage}
\eenu
\vs\vs

{\bf II PARTE. Desarrollo.} Resuelva los siguientes problemas con orden y claridad. En el espacio disponible deben aparecer todos los procedimientos que justifican la respuesta. (Total 20 puntos). \vs\vs

\benu
\item Dé un ejemplo de una terna numérica que represente las medidas de un triángulo escaleno y compruébelo mediante el uso de la Desigualdad Triangular. \hfill{\bf (5 puntos)}

\pagebreak

\item En el siguiente plano cartesiano grafique el cuadrilátero que se forma con los puntos dados. Además escriba en la tabla las coordenadas de 3 puntos interiores y 3 puntos exteriores. {\bf (10 puntos, un punto cada par ordenado).}
\vp

\begin{multicols}{2}
\benu
\item $M(-4,-3)$
\item $A(-2,3)$
\item $T(5,3)$
\item $E(1,-2)$
\eenu

\begin{tabular}{|c|c|}
\hline
\bf Interiores & \bf Exteriores \\
    \hline
    & \\
    \hline
    & \\
    \hline
    & \\
    \hline
\end{tabular}

\end{multicols}


\bc
\definecolor{cqcqcq}{rgb}{0.753,0.753,0.753}
\begin{tikzpicture}[line cap=round,line join=round,>=triangle 45,x=1.0cm,y=1.0cm]
\draw [color=cqcqcq,, xstep=1.0cm,ystep=1.0cm] (-5.78,-5.4) grid (5.66,5.48);
\draw[->,color=black] (-5.78,0.) -- (5.66,0.);
\foreach \x in {-5,-4,-3,-2,-1,1,2,3,4,5}
\draw[shift={(\x,0)},color=black] (0pt,2pt) -- (0pt,-2pt) node[below] {\footnotesize $\x$};
\draw[->,color=black] (0.,-5.4) -- (0.,5.48);
\foreach \y in {-5,-4,-3,-2,-1,1,2,3,4,5}
\draw[shift={(0,\y)},color=black] (2pt,0pt) -- (-2pt,0pt) node[left] {\footnotesize $\y$};
\draw[color=black] (0pt,-10pt) node[right] {\footnotesize $0$};
\clip(-5.78,-5.4) rectangle (5.66,5.48);
\end{tikzpicture}
\ec
\vs\vs

\item ¿Cuáles son las coordenadas del punto medio del segmento determinado por los puntos $(-4,7)$ y $(12,-5)$? \hfill{\bf (5 puntos)} \vs\vs\vs\vs\vs\vs\vs\vs\vs


\eenu

\end{document}


